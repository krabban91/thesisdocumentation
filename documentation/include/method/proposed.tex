% !TEX root = ..\..\main.tex
\section{Proposed model}
\label{sec:method:proposed}
This thesis presents a model that uses relevance feedback and CBIR in order to categorize \todo{a search space of}unlabeled data in an iterative and a more efficient manner. The material that has been \todo{verified or recategorized }recategorized by the user through relevance feedback in previous iterations can be used by the model to present better matches in future iterations. The unlabeled search space will in other words shrink as the labeled set for training will grow.

As mentioned in Section \ref{sec:method:rel_approaches}, the proposed model slightly deviates from other setups. Yet, the general structure is the same as most CBIR systems use. The proposed model consists of three modules; one for matching, one for feature extraction and one that handles relevance feedback. When a search iteration is initiated the matching module fetches a training set and information about the current search space from the feature extraction module, makes elaborate predictions about the material and passes the most accurate information to the relevance feedback module. The relevance feedback module processes the information, requests feedback from the user, passes the ground truths on to the feature extraction model, that updates the search space with new information, and then terminates the iteration which allows a new one to\todo{to be started} start. \todo{The}This communication between the modules during a search iteration can be seen in Figure \ref{fig:meth:proposed:iteration}. 

\singlefigure
{figure/proposed_model.PNG}
{The system consists of three modules; matching, relevance feedback and feature extraction. Here the workflow for the model during a single search iteration is presented.}
{fig:meth:proposed:iteration}
{0.5}

% !TEX root = ..\..\..\main.tex

\subsection{Relevance feedback module}
\label{sec:method:proposed:rf}
There are numerous ways of using relevance feedback in order to improve CBIR and to categorize material. In Section \ref{sec:theory:relfeed} the different ways of relevance feedback are divided into three categories and they are referred to as explicit, implicit and blind feedback. 
In order to make elaborate guesses the model needs to have validated data in its training set and as mentioned in Chapter \ref{chapter:intro} all the case material has to be handled by an investigator in order to build a case. The model has therefore been designed to use explicit feedback in the end of each search iteration.

The feedback that the model receives from the user gives the module information about which images that were correctly categorized and which images that were falsely categorized as negatives and positives. \todo{does the sentence need to be elaborated?}This information could be used to prevent similar mistakes to happen when categorizing the same images. This will however not happen because these images can from this point be used in the training set instead of found in the unlabeled search space and will thereby not be falsely categorized again. Due to the fact that the model is designed to deplete the search space of relevant images, the only information that is drawn from relevance feedback in this model is the knowledge of which images that are relevant and which are not\todo{and which are not, omitt? The one comes with the other} for the specific case. This knowledge is passed on to the feature extraction module and the search iteration is then terminated as seen in Figure \ref{fig:meth:proposed:rf_module}.

\singlefigurenear
{figure/relevance_feedback_module.PNG}
{The relevance feedback module is the intermediator of the user and the rest of the model. When the feedback is given by the user the search iteration can be terminated.}
{fig:meth:proposed:rf_module}
{0.5}

The information that the relevance feedback module receives from the matching module is overly simplified to reduce calculations. The material that is received is sorted to have the most relevant images first and the least relevant images last. The material is however only labeled as relevant and non-relevant. But to present all the material at once would be overwhelming for the user and the model would not need to learn iteratively. 
To present the \emph{top-k} images is a common method within CBIR, and the number of images that are presented could make it easier for the user to oversee the material. The number 20 was arbitrarily chosen and was empirically manageable. An extension to this \emph{top-20} approach in order to improve classification was to also to present some images from the other extreme; the \emph{bottom-k} images. With the intent to avoid drowning the user with information it sufficed with 5 images. Resulting in that the user could quickly give feedback to 25 images in total every iteration\todo{we mention later that we had a trial and error for different setups, is this in line with this text?}. 

% !TEX root = ..\..\..\main.tex
\subsection{Matching module}
\label{sec:meth:proposed:matching}
As a search iteration is initiated the matching module begins with retrieving training data from the feature extraction module. This training data consists of relevant and non-relevant images that can be used to fit the classifier of the model. When the classifier is set up, the search space can be retrieved from the feature extraction module and then be explored. The search space is processed in one batch\todo{works?} at a time to avoid performing predictions on more images than necessary. When the exploration of material has resulted in a sufficient amount of material, the material is sorted from most to least relevant and then passed on to the relevance feedback module with labels of the material being relevant or not. 
In Figure \ref{fig:meth:proposed:matching} there is a visualization of the workflow of the matching module during a search iteration. 

\singlefigure
{figure/matching_module.PNG}
{The matching module performs the initial work of a search iteration. At this point the feature extraction module provides with a training set and details about the search space.}
{fig:meth:proposed:matching}
{0.5}

% !TEX root = ..\..\..\..\main.tex
\subsubsection{Classifier}
\label{sec:method:proposed:matching:classifier}
As mentioned in Section \ref{sec:intro:delimitations}, there was no plan to compare how well different classifiers would perform in this thesis. 
But since having many dimensions can result in more general predictions, a classifier that scales well is to prefer. A classifier that is capable of handling a high number of dimensions is the support vector machine (SVM), see Section \ref{sec:mltheory:svm}. 
Comparative studies such as \cite{IRJET2017classificationMethods}, \cite{SMMR2016comparisionClassificationMethods} and \cite{Informatica2007revClassification} have deemed SVMs as classifiers that continuously show good results in different implementations. 

In the field of CBIR there are myriads of different feature descriptors that are used and the more feature descriptors one can combine, the more general the classifier can become. As mentioned in Section \ref{sec:intro:delimitations} the number of different feature descriptors used in this thesis is limited to five. This solely because adding or removing feature descriptors could improve performance in a general sense, but there was not enough time to cover the subject. In order to combine these five feature descriptors a tree of SVMs were created; one for every feature descriptor and one that treats the output of the different SVMs as input. Read more about the classifier structure \emph{Deep SVM} in Section \ref{sec:deepSVM}. 

As mentioned in Section \ref{sec:method:rel_approaches}, most CBIR systems can quantify certainty of relevance by using a distance function in order to sort the material from most to least relevant. SVMs are geometrical tools that create a decision boundary in some dimensional space to split the categories so they can be categorized depending on which side they are of the decision boundary. Due to the fact that the relevance feedback module expects the material to be sorted the certainty of the classification needs to be quantified. Instead of using the \emph{sign} function to determine the category of an image, the matching module can use the distance between each data point and the decision boundary, a quantification method that is mentioned in Section \ref{sec:mltheory:svm}, to see how probable it is that some image belongs to a certain category. This gives the matching module the possibility to sort the data from the most to the least relevant and the categories of the data points can still be predicted by using the sign function later on.



% !TEX root = ..\..\..\..\main.tex
\subsubsection{Training data}
\label{sec:method:proposed:matching:training}
The training data used to fit the classifier is mainly intended to have been categorized recently by a user. As mentioned in Section \ref{sec:method:proposed} the search space (the  unlabeled material) shrinks as the labeled set grows for every iteration. The labeled data is used to create a training set in order to fit the classifier and make new predictions. 
In the first iteration however, there is no labeled set to work with and therefore no training set for the classifier. When the classifier can not be fit the exploring of the search space is not possible. Instead of locking the workflow of the proposed model the matching module instead selects some material at random (without any predictions) to pass on to the relevance feedback module which creates some labeled data to use in future iterations.

In order to fit an \emph{SVM}\todo{check these emphasises} at least one relevant and one irrelevant data point is necessary and in order to fit a \emph{Deep SVM} it is necessary to have two relevant and two irrelevant data points, due to how it is trained (see Section \ref{sec:deepSVM}). When selecting material from the search space at random there is no guarantee that enough material is sampled in order to fit the classifier correctly within a certain number of iterations. 
To work around this issue the possibility to install an initially predefined training set was introduced. The initial training set is used in combination with the labeled data, that will slowly grow with every search iteration. A justification to add such a training set to the model is that in most cases when a person knows what to look for it has some previous knowledge of how such material would appear. When having a predefined training set with at least 2+2 relevant and irrelevant images, it is always possible to fit the classifier and the search space will be explored. \todokahl{Ofullständing mening}Resulting in actual predictions and better odds of finding more relevant material to improve the training set even more.

Not having enough training data results in the incapability of fitting the classifier correctly, but having too much training data would force the classifier to take too much time to fit the best possible decision boundary for the data. Since the size of a search space could theoretically be infinitely large, so would the labeled set would also go towards infinity as the number of iterations increases. To prevent the labeled set to become to large an upper size limit was set to 500 data points. No evaluation or research was put into this number. Instead it was selected by the intuition of having a training set of that size (500 data points) it should be possible to present a small portion of material with some certainty. Having an upper limit of training material adds the possibility of two things: The time spent training the classifier is done in the same amount of time every iteration and by sampling a subset from a large labeled set allows the decision boundary to shift in-between iteration.


% !TEX root = ..\..\..\..\main.tex
\subsubsection{Exploring search space}
\label{sec:method:proposed:matching:search}
After the search space has been received from the feature extraction module the exploration loop can begin. In this loop the classifier of\todo{in} the matching module is used to categorize the material and calculate \todo{the}their distances from \todo{the data points to }the decision boundary. As mentioned in Section \ref{sec:meth:proposed:matching}, this is done in batches. The batches of the search space are selected at random, i.e. a subset is \todo{is sampled from the search space}easily sampled. The selection process is not covered by the scope of the thesis and is therefore performed as a random search. When the\todo{a} batch of material has been selected the feature extraction module provides with the feature descriptors for the material in the batch. The material is run through the classifier and receives a calculated distance from the classifier. 

In order to avoid continue\todo{remove ``continue''} sampling batches until the entire search space is depleted\todo{every search iteration}, three stopping conditions were introduced. The first rule was implemented to eliminate the risk of an infinite sampling loop. If a sampled batch only would contain material that already has been run through the classifier during the same iteration, the search is over. This stopping condition only exists because of how the batches are selected from the search space and will therefore not be evaluated in the same manner as the other two. The second rule is called \emph{Early stopping}: \todo{if => If}if no relevant images seem to be found, stop the iteration and make the relevance feedback module present the least relevant images. When 200 data points has\todokahl{have} been passed through the classifier and none of them are categorized as relevant the material with the largest distance from the decision boundary is tallied up. Finally, the third rule is called \emph{Threshold}: \todo{meaning => Meaning}meaning that a number of images has to be further away from the decision boundary than a certain value. This value is initially set to 1 until atleast one image in a sampled batch already has been passed through the classifier during the same search iteration. As soon as this occurs the limit changes to a function value depending on the search space size and how many unique images that has been sampled during the search iteration. The entire calculation for the \emph{Threshold} stopping condition value is seen in Equation \ref{eq:method:threshold}\todo{bake the equation}.

\begin{equation}
\label{eq:method:threshold}
\begin{split}
\textnormal{Threshold} = & \left\{
	\begin{matrix}
	\textnormal{unique images only}& 1\\
	\textnormal{otherwise} & \tfrac{n - xlog(x)}{n}
	\end{matrix}
\right .\\
\textnormal{where } n=& \textnormal{ search space size}\\
x =& \textnormal{ numbler of sampled images}
\end{split}
\end{equation}

\todo{in the equation: add ``. '' after sampled images.}

% !TEX root = ..\..\..\main.tex
\subsection{Feature extraction module}
\label{sec:method:proposed:features}
The feature extraction module is really \todo{really not a } a part of the workflow during a search iteration. It is, however, a necessary supporting module that centralizes the control of information regarding search space, predefined training sets and the feature descriptors that are extracted. It provides the matching module with information about the search space \todo{space during search iterations and delivers}for the current iteration and delivers a training set in order for the classifier to work. The feature extraction module updates the search space with ground truths when the relevance feedback has been received and makes sure that the presented data can be used as a part of the training set.

When the matching module is exploring the search space \todo{``,''}the feature extraction module ensures that the feature descriptors of material that is about to be classified is extracted. The process that occurs in the background is visualized in Figure \ref{fig:meth:proposed:feature}. If the feature descriptors for a datapoint\todo{data point} is not extracted, the module extracts them and stores them in the databases for future use as well. As mentioned in Section \ref{sec:intro:delimitations}, only 5\todo{five different feature descriptors} feature descriptors are extracted and used. \todo{why these.}
\TODO{Why these descriptors.}
The following Sections covers how these feature descriptors are extracted. 


\singlefigure
{figure/feature_extraction_module_both.PNG}
{The feature extraction module centralizes the control of information sent to the other modules in the proposed model. The feature extraction module provides data to the matching module and updates the search space with information that is passed on by the relevance feedback module.}
{fig:meth:proposed:feature}
{0.9}
\todo{in the figure: ``sent to the other modules of the proposed''}
% !TEX root = ..\..\..\..\main.tex
\subsubsection{Histogram of oriented gradients}
\label{sec:meth:featextr:hog}

The histogram of oriented gradients (HOG) are implemented using the algorithms developed by \texttt{scikit-learn} \cite{scikitlearn}. The HOG is presented with an image that is converted into a single color channel, gray in this case. In the installment from scikit a number of parameters need to be set. The parameters are the following: The number of pixels per cell and cells per block are selected to determine how large the pooling squares become. The number of orientation bins is selected as well to set how many possible gradients that should be in the output for each block as described in Section \ref{sec:hog_feature}. In this implementation 8 orientations were used, with 32-by-32 pixels per cell and 4-by-4 cells per block. When the convolution is complete the final step is to collect the data from all the blocks and create a feature vector
% !TEX root = ..\..\..\..\main.tex
\subsubsection{Global color histogram}
\label{sec:meth:featextr:gch}

The Global color histogram is implemented as proposed in \cite{wang2015new}, with a few modifications. First\todo{not modifications} the color range of the image is changed to Hue, Saturation and Value (HSV). This color space is covered in Section \ref{sec:HSV_theory}. Each of the color channels is then summarized for their individual different \todo{strange sentence. }number to sort the intensities of the values in a number of pre-determined bins. These numbers can be set to different values based on utility\todo{utility?}. In this project they are set as 24, 12, 6 respectively, instead of the 8, 4 and 2 which are used in Wang et al. implementation\todo{cite again. }. These bins \todo{Big leap. The reader needs help to go from 24,12,6 to 24*12*6=1728.}are divided by the total number of interest points, or maximum possible value to get the values in the range of the SVM as;
\begin{equation}
H(i) = \frac{n_i}{N} \mbox{ where } (i = 1, 2, ..., 1728),
\end{equation}
where $n_i$ number of interest points, N is the total number of interest points and i\todo{i is explained here and in equation. } indicates the possible combinations of the bins. $H(i)$ is used to create one of the feature descriptors used by the classifier described in Section \ref{sec:method:proposed:matching:classifier}.

%To generate the Color histogram first the image is changed into HSV color spectra to have a color range that is more akin to the human perception of its environment. Each of the components of HSV are divided into a number of bins in each of the three representations where the number of values in the different ranges are summed together. To normalise the data and make it suggestible to the SVM the number of interest point in each bin is divided with the total amount of interest points in the image. The resulting values form the feature vector of the Color Histogram. 


% !TEX root = ..\..\..\..\main.tex
\subsubsection{Wavelet transform}
\label{sec:meth:featextr:wt}
The Haar wavelet transform\todo{sentence} compared with other wavelets still has a good performance at cheap computational cost \todo{flip order here. good performance, cheap... compared with other wavelets}explained in Section \ref{sec:wlt_feature}. In the proposed model 3 levels of transformation is used, resulting in a $2^3 = 8$ dimensions and a feature vector of length 144. At the end the values are normalized and placed in an array to form the feature vector, the implementation is based on the one featured proposed in \cite{wang2015new}.\todo{further explaining)}

% !TEX root = ..\..\..\..\main.tex

\subsubsection{Convolutional neural network activations}
\label{sec:meth:featextr:cnn}

%To be able to use the VGG net which is implemented in this system the images need to be of size $256*256$ pixels. 
In the proposed model a pretrained neural network is used. The network is the 16-layer network used by the VGG team in the ILSVRC-2014 competition \cite{simonyan2014very}, also called VGG-16. The network is presented with a $256\times256$ image as input which is processed by the entire network. The feature descriptors are extracted from the activations of the last fully-connected with 4096 neurons, as per Section \ref{sec:vgg_feature}. The vectors are concatenated into a feature vector presentable to the SVM described in Section \ref{sec:mltheory:svm}. 

% !TEX root = ..\..\..\..\main.tex
\subsubsection{Edge detection histogram}
\label{sec:meth:featextr:edge}
\TODO{How? Does this work? is the result an edge image? Is the result a vector of numbers?}
The Sobel edge detector is implemented with four directions instead of two that is more commonly seen in CBIR. The directions are the four major cardinal directions without their respective opposites, thus having the directions seen in the image\todo{images? First mention of image here.}. In addition a Gaussian center blur is added to create a overview feel to the edges. The pixels are put into different numbers of bins that represent the quantitative states of the pixel counts of the images. Also the number of places where no edges\todo{...edges are near are numbered...?} are near are numbered into one last bin\todo{rephrase}. In addition to this the number of edge points in each orientation are presented in one final vector. The edges found in the image are binned as detailed in Section \ref{sec:sob_feature}.


%Is a feature extractor based on the idea of edge detection. The operator is build by creating distinct kernels which are convolved with the image to calculate an approximation of derivatives. For the original Sobel operator this is done on the vertical and horizontal changes. In the implementation the diagonals are also used with a shifted version of the original kernel. This results in defining the strong differences, edges in all direction and in addition a gaussian center blur are work through to create a overview feel to the edges. The pixels that have values are put into different number of bins that represent the quantitative states of the images pixel counts. Also the number of places where no edges are near are numbered into one last bin. In addition to this the number of edge points in each orientation are presented in one final vector string. 




