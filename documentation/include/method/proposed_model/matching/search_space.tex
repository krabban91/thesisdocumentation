% !TEX root = ..\..\..\..\main.tex
\subsubsection{Exploring search space}
\label{sec:method:proposed:matching:search}
After the search space has been received from the feature extraction module the exploration loop can begin. In this loop the classifier of\todo{in} the matching module is used to categorize the material and calculate \todo{the}their distances from \todo{the data points to }the decision boundary. As mentioned in Section \ref{sec:meth:proposed:matching}, this is done in batches. The batches of the search space are selected at random, i.e. a subset is \todo{is sampled from the search space}easily sampled. The selection process is not covered by the scope of the thesis and is therefore performed as a random search. When the\todo{a} batch of material has been selected the feature extraction module provides with the feature descriptors for the material in the batch. The material is run through the classifier and receives a calculated distance from the classifier. 

In order to avoid continue\todo{remove ``continue''} sampling batches until the entire search space is depleted\todo{every search iteration}, three stopping conditions were introduced. The first rule was implemented to eliminate the risk of an infinite sampling loop. If a sampled batch only would contain material that already has been run through the classifier during the same iteration, the search is over. This stopping condition only exists because of how the batches are selected from the search space and will therefore not be evaluated in the same manner as the other two. The second rule is called \emph{Early stopping}: \todo{if => If}if no relevant images seem to be found, stop the iteration and make the relevance feedback module present the least relevant images. When 200 data points has\todokahl{have} been passed through the classifier and none of them are categorized as relevant the material with the largest distance from the decision boundary is tallied up. Finally, the third rule is called \emph{Threshold}: \todo{meaning => Meaning}meaning that a number of images has to be further away from the decision boundary than a certain value. This value is initially set to 1 until atleast one image in a sampled batch already has been passed through the classifier during the same search iteration. As soon as this occurs the limit changes to a function value depending on the search space size and how many unique images that has been sampled during the search iteration. The entire calculation for the \emph{Threshold} stopping condition value is seen in Equation \ref{eq:method:threshold}\todo{bake the equation}.

\begin{equation}
\label{eq:method:threshold}
\begin{split}
\textnormal{Threshold} = & \left\{
	\begin{matrix}
	\textnormal{unique images only}& 1\\
	\textnormal{otherwise} & \tfrac{n - xlog(x)}{n}
	\end{matrix}
\right .\\
\textnormal{where } n=& \textnormal{ search space size}\\
x =& \textnormal{ numbler of sampled images}
\end{split}
\end{equation}

\todo{in the equation: add ``. '' after sampled images.}