% !TEX root = ..\..\..\..\main.tex
\subsubsection{Sobel edge detection histogram}
\label{sec:meth:featextr:edge}
\TODO{How? Does this work? is the result an edge image? Is the result a vector of numbers?}
The Sobel edge detector is implemented with 4\todokahl{skriv ``four''. Direction edge detection? Kanske bättre med directions?} direction edge detection instead of the usual 2\todo{two}. The direction is as the 4\todo{four} major cardinal directions without the opposites, thus having the directions seen in the image\todo{images? First mention of image here.}. In addition a Gaussian center blur is added to create a overview feel to the edges. The pixels that have values are put into different number\todo{numbers} of bins that represent the quantitative states of the \todo{pixel counts of the images}images pixel counts. Also the number of places where no edges\todo{...edges are near are numbered...} are near are numbered into one last bin. In addition to this the number of edge points in each orientation are presented in one final vector string\todo{vector string?}. The edges found in the image are binned as detailed in Section \ref{sec:sob_feature}.


%Is a feature extractor based on the idea of edge detection. The operator is build by creating distinct kernels which are convolved with the image to calculate an approximation of derivatives. For the original Sobel operator this is done on the vertical and horizontal changes. In the implementation the diagonals are also used with a shifted version of the original kernel. This results in defining the strong differences, edges in all direction and in addition a gaussian center blur are work through to create a overview feel to the edges. The pixels that have values are put into different number of bins that represent the quantitative states of the images pixel counts. Also the number of places where no edges are near are numbered into one last bin. In addition to this the number of edge points in each orientation are presented in one final vector string. 


