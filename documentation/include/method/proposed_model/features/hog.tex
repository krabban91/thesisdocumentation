% !TEX root = ..\..\..\..\main.tex
\subsubsection{Histogram of oriented gradients}
\label{sec:meth:featextr:hog}

The histogram of oriented gradients (HOG) are implemented using scikit-learns algorithm \todo{using the algorithms developed by \emph{scikit-learn}}\cite{scikitlearn}. The function\todo{the HOG} is presented with an image \todo{an image that is converted into a}changed into a single color channel, gray is used\todo{remove ``is used''} in this case, as well as \todo{as some parameters}parameters telling the function how thorough\todo{thoroughly} it should \todo{work though => process}work though the image. \todo{The parameters are the following: The number of}Number of pixels per cell and cells per block are selected to determine how large the \todo{resulting square shall be => pooling squares become}resulting square shall be. \todo{As well as the number of}The number of orientation bins are selected to set how many possible gradients \todo{that}should be in the output for each block as described in Section \ref{sec:hog_feature}. In this implementation 8 orientations were used, with 32-by-32 pixels per cell and 4-by-4 cells per block. 