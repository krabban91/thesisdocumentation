% !TEX root = ..\..\..\..\main.tex
\subsubsection{Histogram of oriented gradients}
\label{sec:meth:featextr:hog}

The histogram of oriented gradients (HOG) are implemented using the algorithms developed by \emph{scikit-learn} \cite{scikitlearn}. The HOG is presented with an image that is converted into a single color channel, gray in this case. In the installment from scikit a number of parameters need to be set. The parameters are the following: The number of pixels per cell and cells per block are selected to determine how large the pooling squares become. The number of orientation bins is selected as well to set how many possible gradients that should be in the output for each block as described in Section \ref{sec:hog_feature}. In this implementation 8 orientations were used, with 32-by-32 pixels per cell and 4-by-4 cells per block. When the convolution is complete the final step is to collect the data from all the blocks and create a feature vector