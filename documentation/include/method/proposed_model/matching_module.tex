% !TEX root = ..\..\..\main.tex
\subsection{Matching module}
\label{sec:meth:proposed:matching}
As a search iteration is initiated the matching module begins with retrieving training data from the feature extraction module. This training data consists of relevant and non-relevant images that can be used to fit the classifier of the model. When the classifier is set up, the search space can be retrieved from the feature extraction module and then be explored. The search space is processed in batches\todo{one batch at the time} at a time to avoid performing predictions on more images than necessary. When the exploration of material has resulted in a sufficient amount of material, the material is sorted from most to least relevant and then passed on to the relevance feedback module with labels of the material being relevant or not. 
In Figure \ref{fig:meth:proposed:matching} there is a visualization of the workflow of the matching module during a search iteration. 

\singlefigure
{figure/matching_module.PNG}
{The matching module performs the initial work of a search iteration. At this point the feature extraction module provides with a training set and details about the search space.}
{fig:meth:proposed:matching}
{0.5}

% !TEX root = ..\..\..\..\main.tex
\subsubsection{Classifier}
\label{sec:method:proposed:matching:classifier}
As mentioned in Section \ref{sec:intro:delimitations}, there was no plan to compare how well different classifiers would perform in this thesis. 
But since having many dimensions can result in more general predictions, a classifier that scales well is to prefer. A classifier that is capable of handling a high number of dimensions is the support vector machine (SVM), see Section \ref{sec:mltheory:svm}. 
Comparative studies such as \cite{IRJET2017classificationMethods}, \cite{SMMR2016comparisionClassificationMethods} and \cite{Informatica2007revClassification} have deemed SVMs as classifiers that continuously show good results in different implementations. 

In the field of CBIR there are myriads of different feature descriptors that are used and the more feature descriptors one can combine, the more general the classifier can become. As mentioned in Section \ref{sec:intro:delimitations} the number of different feature descriptors used in this thesis is limited to five. This solely because adding or removing feature descriptors could improve performance in a general sense, but there was not enough time to cover the subject. In order to combine these five feature descriptors a tree of SVMs were created; one for every feature descriptor and one that treats the output of the different SVMs as input. Read more about the classifier structure Deep SVM in Section \ref{sec:deepSVM}. 

As mentioned in Section \ref{sec:method:rel_approaches}, most CBIR systems can quantify certainty of relevance by using a distance function in order to sort the material from most to least relevant. SVMs are geometrical tools that create a decision boundary in some dimensional space to split the categories so they can be categorized depending on which side they are of the decision boundary. Due to the fact that the relevance feedback module expects the material to be sorted the certainty of the classification needs to be quantified. Instead of using the \emph{sign} function, Equation (\ref{eq:svm_test}), to determine the category of an image, the matching module can use the distance between each data point and the decision boundary, a quantification method that is mentioned in Section \ref{sec:mltheory:svm}, to see how probable it is that some image belongs to a certain category. This gives the matching module the possibility to sort the data from the most to the least relevant and the categories of the data points can still be predicted by using the sign function later on.



% !TEX root = ..\..\..\..\main.tex
\subsubsection{Training data}
\label{sec:method:proposed:matching:training}
The training data used to fit the classifier is mainly intended to have been classified \todo{been categorized recently by the user.}by the user recently. As mentioned in Section \ref{sec:method:proposed} the search space (\todo{the } unlabeled material) shrinks as the labeled set grows for every iteration. The labeled data is used to create a training set in order to fit the classifier and make new predictions. 
In the first iteration however, there is no labeled set to work with and therefor\todo{therefore} no training set for the classifier. When the classifier can not be fit the exploring of the search space is not possible. Instead of locking the workflow of the proposed model the matching module instead \todo{selects some material at random}randomly selects some material (without any predictions) to pass on to the relevance feedback module and in the following iterations there will be some labeled data to use\todo{which creates some labeled data to use in the following iterations?}.

In order to fit an\todo{-an +a} \emph{SVM}\todo{check these emphasises} at least one relevant and one irrelevant data point is necessary and in order to fit a \emph{Deep SVM} it is necessary to have two relevant and two irrelevant data points\todo{, due to how it is trained (see Section \ref{sec:deepSVM}).}\todo{does this require further explanation?}. When selecting material from the search space at random there is no guarantee that enough material is sampled in order to fit the classifier correctly within a certain number of iterations. 
To work around this issue the possibility to install an initially \emph{predefined training set} was given\todo{created}. The initial training set is used in combination with the labeled data\todo{add ``,''} that will slowly grow with every search iteration. A justification to add such a training set to the model is that in most cases when a person knows what to look for it has some previous knowledge of how such material would appear. It would even be improbable that the user would not have some material at hand ready to be inserted into the model\todo{relevance?}. When having a predefined training set with at least 2+2 relevant and irrelevant images, it is always possible to fit the classifier and the search space will be explored. \todokahl{Ofullständing mening}Resulting in actual predictions and better odds of finding more relevant material to improve the training set even more.

Not having enough training data results in the incapability of fitting the classifier correctly, but having too much training data would \todo{force}make the classifier to take too much time to fit the best \todo{possible }decision boundary for the data\todo{mention overfitting?}. Since the size of a search space could\todo{be} infinitely large, so could labeled set be after a large number of iterations\todo{rephrase}. To prevent the labeled set to become to large an upper size limit was set to 500 data points. No evaluation or research was put into this number. Instead it was selected by the intuition of having a training set of that\todo{vageu size?} size it should be possible to present a small portion of material with some certainty. Having an upper limit of training material adds the possibility of two things: The time spent training the classifier is done in the same amount of time every iteration and by sampling a subset from a large labeled set allows the decision boundary to shift in\todo{in-between} between iteration\todo{is this essentially good, need elaboration?}.   


% !TEX root = ..\..\..\..\main.tex
\subsubsection{Exploring search space}
\label{sec:method:proposed:matching:search}
After the search space has been received from the feature extraction module the exploration loop can begin. In this loop the classifier of\todo{in} the matching module is used to categorize the material and calculate \todo{the}their distances from \todo{the data points to }the decision boundary. As mentioned in Section \ref{sec:meth:proposed:matching}, this is done in batches. The batches of the search space are selected at random, i.e. a subset is \todo{is sampled from the search space}easily sampled. The selection process is not covered by the scope of the thesis and is therefore performed as a random search. When the\todo{a} batch of material has been selected the feature extraction module provides with the feature descriptors for the material in the batch\todo{flow}. The material is run through\todo{presented to? or maybe works when i look now} the classifier and receives a calculated distance from\todo{by} the classifier. 

In order to avoid continue\todo{remove ``continue''} sampling batches until the entire search space is depleted\todo{every search iteration}, three stopping conditions were introduced. The first rule was implemented to eliminate the risk of an infinite sampling loop. If a sampled batch only would contain material that already has been run through the classifier during the same iteration, the search is over. This stopping condition only exists because of how the batches are selected from the search space and will therefore not be evaluated in the same manner as the other two. The second rule is called \emph{Early stopping}: \todo{if => If}if no relevant images seem to be found, stop the iteration and make the relevance feedback module present the least relevant images. When 200 data points has\todokahl{have} been passed through the classifier and none of them are categorized as relevant the material with the largest distance from the decision boundary is tallied up. Finally, the third rule is called \emph{Threshold}: \todo{meaning => Meaning}meaning that a number of images has to be further away from the decision boundary than a certain value. This value is initially set to 1 until atleast one image in a sampled batch already has been passed through the classifier during the same search iteration. As soon as this occurs the limit changes to a function value depending on the search space size and how many unique images that has been sampled during the search iteration. The entire calculation for the \emph{Threshold} stopping condition value is seen in Equation \ref{eq:method:threshold}\todo{bake the equation}.

\begin{equation}
\label{eq:method:threshold}
\begin{split}
\textnormal{Threshold} = & \left\{
	\begin{matrix}
	\textnormal{unique images only}& 1\\
	\textnormal{otherwise} & \tfrac{n - xlog(x)}{n}
	\end{matrix}
\right .\\
\textnormal{where } n=& \textnormal{ search space size}\\
x =& \textnormal{ numbler of sampled images}
\end{split}
\end{equation}

\todo{in the equation: add ``. '' after sampled images.}
