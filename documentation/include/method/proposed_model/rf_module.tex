% !TEX root = ..\..\..\main.tex

\subsection{Relevance feedback module}
\label{sec:method:proposed:rf}
There are numerous ways of using relevance feedback in order to improve CBIR and to categorize material. In Section \ref{sec:theory:relfeed} the different ways of relevance feedback are divided into three categories and they are referred to as explicit, implicit and blind feedback. 
In order to make elaborate guesses the model needs to have validated data in its training set and as mentioned in Chapter \ref{chapter:intro} all the case material has to be handled by an investigator in order to build a case. The model has therefore been designed to use explicit feedback in the end of each search iteration.

The feedback that the model receives from the user gives the module information about which images that were correctly categorized and which images that were falsely categorized as negatives and positives. The information of which images that have been categorized as false negatives or positives could be used in order to improve classification in some direction. But since there are many ways of using the relevance feedback and the scope of the thesis is limited, the different methods of using this information were not explored. Due to the fact that the model is designed to deplete the search space of relevant images, the only information that is drawn from relevance feedback in this model is the knowledge of which images that are relevant and non-relevant for the specific case. This knowledge is passed on to the feature extraction module and the search iteration is then terminated as seen in Figure \ref{fig:meth:proposed:rf_module}.

\singlefigurenear
{figure/relevance_feedback_module.PNG}
{The relevance feedback module is the intermediator of the user and the rest of the model. When the feedback is given by the user the search iteration can be terminated.}
{fig:meth:proposed:rf_module}
{0.5}

The information that the relevance feedback module receives from the matching module is overly simplified to reduce calculations. The material that is received is sorted to have the most relevant images first and the least relevant images last. The material is however only labeled as relevant and non-relevant. But to present all the material at once would be overwhelming for the user and the model would not need to learn iteratively. 
To present the \emph{top-k} images is a common method within CBIR, and the number of images that are presented could make it easier for the user to oversee the material. The number 20 was arbitrarily chosen and was empirically manageable. An extension to this \emph{top-20} approach in order to improve classification was to also to present some images from the other extreme; the \emph{bottom-k} images. With the intent to avoid drowning the user with information it sufficed with 5 images. Resulting in that the user could quickly give feedback to 25 images in total every iteration. The setting of which images and how many of each group of images that are presented each iteration was not easy to set. This decision paned out to be  made using measurements and is described in Section \ref{sec:method:eval:param}. 
