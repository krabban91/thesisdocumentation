% !TEX root = ..\..\..\main.tex
\subsection{Feature extraction module}
\label{sec:method:proposed:features}
The feature extraction module is really \todo{really not a } a part of the workflow during a search iteration\todo{rephrase bur maybe not needed after change}. It is, however, a necessary supporting module that centralizes the control of information regarding search space, predefined training sets and the feature descriptors that are extracted. It provides the matching module with information about the search space \todo{space during search iterations and delivers}for the current iteration and delivers a training set in order for the classifier to work. The feature extraction module updates the search space with ground truths when the relevance feedback has been received and makes sure that the presented data can be used as a\todo{-a?} part of the training set.

When the matching module is exploring the search space \todo{``,''}the feature extraction module ensures that the feature descriptors of material that is about to be classified is extracted\todo{sounds strange, the feature descriptors handling of material that is... maybe?}. The process that occurs in the background is visualized in Figure \ref{fig:meth:proposed:feature}. If the feature descriptors for a datapoint\todo{data point} is not extracted, the module extracts them and stores them in the databases for future use as well. As mentioned in Section \ref{sec:intro:delimitations}, only 5\todo{five different feature descriptors} feature descriptors are extracted and used. \todo{why these.}
\TODO{Why these descriptors.}
The following Sections\todo{sections} covers how these feature descriptors are extracted. 


\singlefigure
{figure/feature_extraction_module_both.PNG}
{The feature extraction module centralizes the control of information sent to the other modules in the proposed model. The feature extraction module provides data to the matching module and updates the search space with information that is passed on by the relevance feedback module.}
{fig:meth:proposed:feature}
{0.9}
\todo{in the figure: ``sent to the other modules of the proposed''}
% !TEX root = ..\..\..\..\main.tex
\subsubsection{Histogram of oriented gradients}
\label{sec:meth:featextr:hog}

The histogram of oriented gradients (HOG) are implemented using scikit-learns algorithm \todo{using the algorithms developed by \emph{scikit-learn}}\cite{scikitlearn}. The function\todo{the HOG} is presented with an image \todo{an image that is converted into a}changed into a single color channel, gray is used\todo{remove ``is used''} in this case, as well as \todo{as some parameters}parameters\todo{elaborate} telling the function how thorough\todo{thoroughly} it should \todo{work though => process}work though the image. \todo{The parameters are the following: The number of}Number of pixels per cell and cells per block are selected to determine how large the \todo{resulting square shall be => pooling squares become}resulting square shall be\todo{what square}. \todo{As well as the number of}The number of orientation bins are selected to set how many possible gradients \todo{that}should be in the output for each block as described in Section \ref{sec:hog_feature}. In this implementation 8 orientations were used, with 32-by-32 pixels per cell and 4-by-4 cells per block\todo{better finish}. 
% !TEX root = ..\..\..\..\main.tex
\subsubsection{Global color histogram}
\label{sec:meth:featextr:gch}

The Global color histogram is implemented as proposed in \cite{wang2015new}, with a few modifications. First\todo{not modifications} the color range of the image is changed to Hue, Saturation and Value (HSV). This color space is covered in Section \ref{sec:HSV_theory}. Each of the color channels is then summarized for their individual different \todo{strange sentence. }number to sort the intensities of the values in a number of pre-determined bins. These numbers can be set to different values based on utility\todo{utility?}. In this project they are set as 24, 12, 6 respectively, instead of the 8, 4 and 2 which are used in Wang et al. implementation\todo{cite again. }. These bins \todo{Big leap. The reader needs help to go from 24,12,6 to 24*12*6=1728.}are divided by the total number of interest points, or maximum possible value to get the values in the range of the SVM as;
\begin{equation}
H(i) = \frac{n_i}{N} \mbox{ where } (i = 1, 2, ..., 1728),
\end{equation}
where $n_i$ number of interest points, N is the total number of interest points and i\todo{i is explained here and in equation. } indicates the possible combinations of the bins. $H(i)$ is used to create one of the feature descriptors used by the classifier described in Section \ref{sec:method:proposed:matching:classifier}.

%To generate the Color histogram first the image is changed into HSV color spectra to have a color range that is more akin to the human perception of its environment. Each of the components of HSV are divided into a number of bins in each of the three representations where the number of values in the different ranges are summed together. To normalise the data and make it suggestible to the SVM the number of interest point in each bin is divided with the total amount of interest points in the image. The resulting values form the feature vector of the Color Histogram. 


% !TEX root = ..\..\..\..\main.tex
\subsubsection{Wavelet transform}
\label{sec:meth:featextr:wt}
The Haar wavelet transform has good performance at a cheap computational cost compared to many other wavelet transforms explained in Section \ref{sec:wlt_feature}. In the proposed model 3 levels of transformation is used, resulting in a $2^3 = 8$ dimensions and a feature vector of length 144. At the end the values are normalized and placed in an array to form the feature vector, the implementation is based on the one featured proposed in \cite{wang2015new}.\todo{further explaining)}

% !TEX root = ..\..\..\..\main.tex

\subsubsection{Convolutional neural network activations}
\label{sec:meth:featextr:cnn}

%To be able to use the VGG net which is implemented in this system the images need to be of size $256*256$ pixels. 
In the proposed model a pretrained neural network is used. The network is the 16-layer network used by the VGG team in the ILSVRC-2014 competition \cite{simonyan2014very}, also called \emph{VGG-16}. The network is presented with a $256\times256$ image as input which is processed by the entire network. The feature descriptors are extracted from the activations of the last fully-connected with 4096 neurons, as per Section \ref{sec:vgg_feature}. The vectors are concatenated into a feature vector presentable to the SVM described in Section \ref{sec:mltheory:svm}. 

% !TEX root = ..\..\..\..\main.tex
\subsubsection{Edge detection histogram}
\label{sec:meth:featextr:edge}
The Sobel edge detector is implemented with four directions instead of two that is more commonly seen in CBIR. The directions are four of the eight major cardinal directions without their respective opposites. In addition a Gaussian center blur is added to create a overview feel to the edges. The pixels are put into different numbers of bins that represent the quantitative states of the pixel counts of the images in their respective orientations. The locations without edges are also formed into a separate bin. The edges found in the image are binned as detailed in Section \ref{sec:sob_feature}. The output is a feature vector of values, describing the shapes shifts present in the image. 


%Is a feature extractor based on the idea of edge detection. The operator is build by creating distinct kernels which are convolved with the image to calculate an approximation of derivatives. For the original Sobel operator this is done on the vertical and horizontal changes. In the implementation the diagonals are also used with a shifted version of the original kernel. This results in defining the strong differences, edges in all direction and in addition a gaussian center blur are work through to create a overview feel to the edges. The pixels that have values are put into different number of bins that represent the quantitative states of the images pixel counts. Also the number of places where no edges are near are numbered into one last bin. In addition to this the number of edge points in each orientation are presented in one final vector string. 



