 % !TEX root = ..\..\..\..\main.tex
\subsubsection{Classifier learning method}
\label{sec:method:eval:param:learning}
Since the proposed model is designed to acquire deeper understanding of a concept for every passing iteration, it is important that what the model learns from is chosen in an appropriate manner. 25 images are presented each iteration and this evaluation is designed to decide how these 25 images should be selected . The best setting in this benchmark is used in the evaluations that are performed later. 
The goal of the evaluation is to find a method that ensures that the model learns the concept as fast as possible but still perform well enough in order to reduce work for the user. 

Four distinct settings were chosen that are representative of how the data can be selected and still processes the dataset in an efficient manner. The different settings of the evaluation are
\begin{enumerate}
	\item \textbf{Top20+Bottom5}: To present the 20 images that the classifier finds the most relevant and the 5 images that the classifier finds the \todo{least relevant}most irrelevant. As mentioned in Section \ref{sec:method:proposed:rf}, this is the how the model is designed to present material.  
	\item \textbf{Top25}: To present the 25 images that the classifier finds the most relevant. 
	\item \textbf{Top20+Middle5}: To present the 20 images that the classifier finds the most relevant and the 5 images that are the closest to the decision boundary of the classifier. 
	\item \textbf{Top5+Bottom20}: To present the 5 images that the classifier finds the most relevant and the 20 images that the classifier finds the least relevant. 
\end{enumerate}
In order to make the different settings deviate as much as possible, the model processed the entire search space every iteration. Thus making the selection of images each iteration more predictable and the measurements of each setting are less diverged in between the five different runs.

