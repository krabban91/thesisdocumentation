% !TEX root = ..\..\..\main.tex
\subsection{Parameter benchmarks}
\label{sec:method:eval:param}
The presentation of the proposed model in Section \ref{sec:method:proposed} covers the most part of how the optimal implementation is designed. What it does not cover is if some of the design decisions are good or not. For example Section \ref{sec:method:proposed:rf} the selection of images to present to the user is mentioned but not elaborated. There are two stopping conditions introduced in Section \ref{sec:method:proposed:matching:search} which need to be evaluated to see how they effect the performance. The five feature descriptors covered in Section \ref{sec:method:proposed:features}, need to be evaluated. Not only in combination but how they perform on their own.
%In Section \ref{sec:method:proposed:features}, that covers the feature extraction module, five different feature descriptors are extracted to be used by the classifier. 
As well as the effect of the training sets mentioned in Section \ref{sec:method:proposed:matching:training}. How much can the behavior of the model change depending on training set sizes.
To summarize, the parameter benchmarks are the following four evaluations:
\begin{itemize}
\item The effect of presenting different sets of images to the user for relevance feedback (Section \ref{sec:method:eval:param:learning}).
\item The effect of pruning the search space during the search iterations (Section \ref{sec:method:eval:param:stopping}).
\item How well the ensemble of learners performs compared to its parts (Section \ref{sec:method:eval:param:features}).
\item The effect of using different kinds of training sets (Section \ref{sec:method:eval:param:training}).
\end{itemize}
\medskip

Before specifying which metrics that are intended to be used in these evaluations, the metrics need to be defined. The two more commonly used measures are (\ref{eq:meth:recall}) and (\ref{eq:meth:precision}).

\begin{equation}
\label{eq:meth:recall}
\textnormal{recall} = \tfrac{\textnormal{True positives}}{\textnormal{True positives } + \textnormal{ False negatives}}
\end{equation}

\begin{equation}
\label{eq:meth:precision}
\textnormal{precision} = \tfrac{\textnormal{True positives}}{\textnormal{True positives } + \textnormal{ False positives}}.
\end{equation}
Often presented in pairs due to the fact that recall can reveal if the matching module presents too many false negatives while precision can indicate if too many false positives are presented. 

A metric that measures the models effectiveness is the F1-measure (\ref{eq:meth:f1measure})

\begin{equation}
\label{eq:meth:f1measure}
\begin{split}
\textnormal{F1-measure} &= 2 \tfrac{precision \times recall}{precision + recall}\\
&= \tfrac{\textnormal{True positives}}{\textnormal{True positives} +\frac{\textnormal{False positives}+\textnormal{False negatives}}{2}}.
\end{split}
\end{equation}

The F1-measure is the harmonic mean of recall and precision \cite{powers2011evaluation}, meaning that recall and precision are equally weighted and that smaller values are punished more than when using a normal average function. 

The final metric to present is the accuracy (\ref{eq:meth:accuracy}) 

\begin{equation}
\label{eq:meth:accuracy}
\textnormal{accuracy} = \tfrac{\textnormal{True positives} + \textnormal{True negatives}}{\textnormal{True positives} + \textnormal{True negatives}+ \textnormal{False positives} + \textnormal{False negatives}},
\end{equation}
which simply put is the total rate of correct predictions. 


Apart from these metrics the evaluations include how long the calculations of an iteration takes, how many images that are passed through the classifier in the matching module and how many of the presented images that are relevant.   

In order to make sure that the search space is explored and categorized correctly and at the same time see how well the model performs at classification, two different methods are used to calculate performance. In addition to the search space an evaluation set, in complete disjunction to the search space, is used. The datasets for the benchmarks and how they are constructed is presented in Section \ref{sec:meth:eval:bench:dataset}. 

The benchmark evaluations are measured in two different ways.
\begin{enumerate}
\item How well the model classifies an evaluation set each iteration. For the evaluation set the following metrics are used: 
	\begin{itemize}
		\item Recall
		\item Precision
		\item F1-Measure
		\item Accuracy
	\end{itemize}   
\item How well the model classifies the images that are presented to the user each iteration. For the search space the following metrics are used:
	\begin{itemize}
		\item Accumulated recall 
		\item Accumulated precision
		\item Accumulated F1-Measure
		\item Accumulated accuracy
		\item Retrieved relevant images
		\item Handled images
		\item Time taken calculating
	\end{itemize}   
\end{enumerate}
\medskip
When measuring the performance of classifying the search space the metrics are the accumulated value of the performance so far during the search. This mean that after iteration 20 the performance is measured on the 500 images that have received an prediction by the proposed.

In order to retrieve a general trend, each evaluation setting is run five separate times and the metrics during these runs are presented with the maximum, the minimum and the geometric mean of each iteration. This allows the graphs that are presented in this section to show how much the metrics varied depending on different factors, such as image selection or how the decision boundary was fitted.

In all benchmarks, except for the one in Section \ref{sec:method:eval:param:features} where the ensemble is evaluated against its parts, the classifier is implemented as proposed in Section \ref{sec:method:proposed:matching:classifier} and will have use all five feature descriptors. In addition; all benchmarks, with the exception of the one described in Section \ref{sec:method:eval:param:training} where training data is evaluated, the matching model will have a predefined training set of 5 relevant images and 50 non-relevant.

The evaluation of the model is intended to be fair and the performance to be measured in an as general way as possible but still possible to perform within a relatively small time frame. The datasets for the evaluation are therefore constructed specifically for this purpose. 

% !TEX root = ..\..\..\..\main.tex
\subsubsection{Datasets for benchmark}
\label{sec:meth:eval:bench:dataset}

Since scenery datasets such as the dataset Places205, presented in Section \ref{sec:theory:dataset:places}, have a broad base of image material and have a large variety of material within the categories, Places205 is perfect for a parameter benchmark. However, due to the size of the dataset, the decision to only use a subset was made. Instead of using all 205 different scenery categories a subset of 23 classes was cherry-picked; all categories with a name starting with the letter B. This gives a $\approx 4.3\%$ chance that a randomly selected image is relevant. But since only 25 images are presented per iteration using the entire subset as a search space would cause the number of iterations to be $\approx 11000$, the search space was reduced a bit further. The search space during the benchmark evaluations consists of 200 images of each category where one of the categories is marked as relevant and the others are not. 
The choice of just using a part of the dataset makes it hard to compare with other implementations of image recognition implemented on this dataset. As well as the fact that the algorithm is designed to be more lightweight and less time-consuming than the more usual approach of using large deep neural networks.

In Section \ref{sec:method:eval:param} an evaluation set is mentioned. The Evaluation set is constructed by using 50 images, that are not represented in the search space, of each category. Resulting in having an evaluation set of 1150 images with the same probability of randomly selecting a relevant image as in the search space.  

In addition to the 250 images of every category used in the search space and the evaluation set, 250 images of every category were sampled in order to have material for different predefined training sets. The different sizes of the predefined training sets vary and how they are evaluated is presented in Section \ref{sec:method:eval:param:training}. The different training sets are simply constructed to have an as broad base on irrelevant images -- sampling some images from all categories that are not relevant -- and the smaller sets of relevant images are subsets of the larger sets of relevant images.  

To not put all eggs in the same basket, the benchmarks have been performed with three different categories marked as relevant in three different evaluations. These three categories are Bar, Baseball field and Bedroom. The three categories does not have very much in common but some of the other categories in the dataset might have some similarities. Having three different categories results in having three different search spaces, three different evaluation sets and three different setups of training sets. The data drawn from these categories will be presented in parallel. 


% !TEX root = ..\..\..\..\main.tex
\subsubsection{Classifier learning method}
\label{sec:method:eval:param:learning}
Since the proposed model is designed to acquire deeper understanding for\todo{of} a concept for \todo{for each => for every passing}each iteration, it is important that \todo{that what the model}the model learns from is chosen in an appropriate manner. 25 images are presented each iteration and this evaluation is designed to decide \todo{how these 25 images should be selected}which setting that is the optimal one. The best setting in this benchmark is used in the evaluations that are performed later. 
The goal of the evaluation is to find a method that ensures that the model learns the concept as fast as possible but still perform well enough in order to reduce work for the user. 

Four distinct settings were chosen that are representative of how the data can be selected and still \todo{processes the dataset in an...}work through the data in an efficient manner. The different settings \todo{of the evaluation are}that were used are
\begin{enumerate}
	\item \textbf{Top20+Bottom5}: To present the 20 images that the classifier finds the most relevant and the 5 images that the classifier finds the \todo{least relevant}most irrelevant. As mentioned in Section \ref{sec:method:proposed:rf}, this is the how the model is designed to be present\todo{presented?} material.  
	\item \textbf{Top25}: To present the 25 images that the classifier finds the most relevant. 
	\item \textbf{Top20+Middle5}: To present the 20 images that the classifier finds the most relevant and the 5 images that are the closest to the decision boundary of the classifier. 
	\item \textbf{Top5+Bottom20}: To present the 5 images that the classifier finds the most relevant and the 20 images that the classifier finds the \todo{least relevant}most irrelevant. 
\end{enumerate}
In order to make the different settings deviate as much as possible, the model \todo{processed}classified the entire search space every iteration. Thus making the selection of images each iteration more predictable and the measurements\todo{add: of each setting} are less diverged in between the five different runs.


% !TEX root = ..\..\..\..\main.tex
\subsubsection{Limiting search space}
\label{sec:method:eval:param:stopping}

Evaluating the entire search space every iteration is not just time consuming but also unnessecary since only 25 images are presented at a time. The stopping conditions presented in Section \ref{sec:method:proposed:matching:search} are therefore evaluated to measure their effect on performance.
As previously mentioned the first stopping condition -- if a sample from the search space only contains material that has been passed through the classifier the same search iteration the exploration is over --  is always in use to prevent the possibility of an infinite loop while exploring. 

The four different settings evaluated in this benchmark are
\begin{enumerate}
	\item \textbf{All images:} To stop after the entire search space is evaluated. 
	\item \textbf{Threshold:} To stop after enough images are classified to be above a decision threshold specified in Equation \ref{eq:method:threshold}. 
	\item \textbf{Early stopping:} To stop if 200 images have been sampled but none are on the positive side of the decision boundary. In this case the relevance feedback module presents the 25 images that are considered the least relevant. 
	\item \textbf{Both rules:} The additive result of using setting 2 and 3.
\end{enumerate} 
\medskip

The intention of the evaluation is to measure the trade-of between correctness in classification of the search space and time spent calculating istances to different data points.

% !TEX root = ..\..\..\..\main.tex
\subsubsection{Feature descriptors}
\label{sec:method:eval:param:features}
To determine how well the classifier performs as an ensemble compared to only using its parts this evaluation has been constructed by using \todo{remove ``or not using''}or not using the first order classifiers either by themselves or in unison. When only usings one classifier in the first order, the second order classifier is given a linearly separable value in one dimension and thereby just passes on the information. This evaluation is intended to see which feature descriptors that work well on which categories in the data set. But the intention is also to evaluate if different combinations of feature descriptors will improve the performance of the model. 
Therefore the evaluation handles 7 different settings: 
\begin{enumerate}
	\item \textbf{HOG:} Only using the first order classifier for the HOG descriptors (see Section \ref{sec:meth:featextr:hog}).
	\item \textbf{GCH:} Only using the first order classifier for the GCH descriptors (see Section \ref{sec:meth:featextr:gch}).
	\item \textbf{WT:} Only using the first order classifier for the \todo{Haar wavelet transform}wavelet transformation descriptors (see Section \ref{sec:meth:featextr:wt}).
	\item \textbf{CNN:} Only using the first order classifier for the neural network activation vectors as descriptors (see Section \ref{sec:meth:featextr:cnn}).
	\item \textbf{Edge:} Only using the first order classifier for the \todo{refer to accordingly with section name}edge detection descriptors (see Section \ref{sec:meth:featextr:edge}).
	\item \textbf{All:} Combining all the first order classifiers as intended and \todo{proposed in section => as in the proposed model described in Section \ref{sec:method:proposed}.}proposed in Section \ref{sec:method:proposed:matching:classifier}. 
	\item \textbf{All-CNN:} Combining all the first order classifiers with the exception of the neural network activation vectors. This setting was added half-way through the evaluation since the setting \textbf{CNN} almost performed as well as the setting \textbf{All}.
\end{enumerate}
% !TEX root = ..\..\..\..\main.tex
\subsubsection{Training data}
\label{sec:method:eval:param:training}

Is it necessary to provide initial training data to the model in order to improve classification correctness \todo{before learning => in order to learn}before learning a specific concept? \todo{And if => If}And if so, how much difference would it make? This evaluation is designed to answer these two questions. A comparison will be made between only having a predefined training set, only using the data that is provided from the user during iterations and a combination of the two. On top of this, the size of the pretrained data set was also evaluated. 

The different settings during the training data evaluation were 
\begin{enumerate}
	\item to train the classifier once in the beginning with given data and use the same classifier until the search space is empty. 
	\item to train the classifier every iteration with data given by relevance feedback together with a predefined training set. 
	\item to train the classifier every iteration with data given by relevance feedback only.
\end{enumerate}
\medskip
The predefined training sets were assembled by different number of relevant and non-relevant images. The different sets consist of
\begin{enumerate}[label=\Alph*.]
	\item 5 relevant and 5 non-relevant images.
	\item 5 relevant and 50 non-relevant images.
	\item 22 relevant and 484 non-relevant images (thus giving it the same relevance ratio as the search space).
	\item 250 relevant and 250 non-relevant images (thus making it contain more relevant images than the search space does).
\end{enumerate}
\medskip
Since there are no good and concise ways to refer to these different settings, they will in this section be referred to by their index in these two lists. The setting that only uses a predefined data set of 5 relevant images and 50 non-relevant images is referred to as \todo{remove emph}\emph{setting 1B} and the setting that solely uses data given by relevance feedback is referred to as \todo{remove emph}\emph{setting 3}.

As mentioned in Section \ref{sec:method:proposed:matching:training} the model uses at most 500 images \todo{taken from the relevance feedback as training data} from the relevance feedback data. Those 500 images are intended to be as close to evenly divided between relevant and non-relevant images as possible. Since the search space consists of 200 relevant images and 4400 non-relevant images the training data in \todo{remove emph}\emph{setting 3} consists of at most 200 relevant images and 300 non-relevant while \todo{remove emph}\emph{setting 2D} will have at most 450 relevant images and 550 non-relevant images in its training set. 
\todo{This since => Since}This since the setting combines the two training sets.
