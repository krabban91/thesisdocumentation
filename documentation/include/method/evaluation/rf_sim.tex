% !TEX root = ..\..\..\main.tex
\subsection{Relevance feedback simulation}
\label{sec:method:rf_simulation}
 relevance feedback is, as the proposed model suggests in Section \ref{sec:method:proposed:rf}, used to help the model to mount the semantic gap. A user peer-reviews \todo{each image => the presented images each iteration}each image and makes corrections where necessary to \todo{to make sure that the entire dataset is labeled correctly.}make the entire data set labeled correctly. But to do this on large data sets is both time consuming and takes quite a toll on the user. There is also no guarantee that a user will label the same image as the same category in two separate settings but when searching for the same material. 
Since the evaluations will \todo{are run}be run on datasets that already are labeled this risk of mistakes can be reduced by instantiating a user simulation. The relevance feedback simulation takes the role of a user that communicates with the relevance feedback module. This way time is saved and no user needs to be present in order for the evaluations to run\todo{be performed}.  

This\todo{The} simulated user will not be misslabeling images because of negligence or exhaustion as a human would. 
A normal user would fail to label material correctly in the same extent as a simulated one would. 
\todo{This sentence does not work in our favour.}However, the error rate for a ``trained'' human is extremely low, comparable to the best neural networks that compete in the \emph{ImageNet Large Scale Visual Recognition Challenge} (ILSVRC), e.g. GoogleNet \cite{ImageNetChallenge}. 
In this case a ``trained'' person refers to a person that is aware of the situation and well versed in classifying images. \todo{Yet, this still means 4\% misclassification compared to 0\%.}
The data sets presented in Section \ref{sec:theory:datasets} are very diverse and in some cases a simulated user would categorize the material differently than an ordinary on\todo{one} would. This is however not a problem rooted in how the simulated user works but how the datasets are designed and categorized to begin with.
