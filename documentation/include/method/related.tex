% !TEX root = ..\..\main.tex
\section{Related approaches}
\label{sec:method:rel_approaches}
The approach of modeling a system presented in this thesis is, \todo{to our knowledge}after checking numerous papers,\todokahl{Man brukar skriva ``to our knowledge''} still untested.
There are however several papers that implement different components of the proposed model.
Most content \todo{content-based}based image retrieval (CBIR) systems use a set of different feature descriptors that are proven to be good at finding equalities or similarities between images. The system is then presented with a single query image in order to find matches in a database. The images in the database are compared to this query image and the similarities are calculated in some way \cite{wang2001simplicity}\cite{subrahmanyam2013modified}\cite{nagaraja2015low}, e.g. calculating the Euclidean distance and sorting the data having the most similar \todo{data points}first.
Other CBIR systems\todo{have} has\todokahl{have} created a feature vector from extracting certain feature descriptors, trained a neural network with a subset of the data and uses\todo{use} the classifier to retrieve images \cite{elalami2014new}. 
There are other implementations where the use of relevance feedback is used in conjunction with the feature descriptors to even further increase performance, in light of the difficulty \todo{to }identify \todo{feature descriptors}descriptors that are good at ``understanding'' concepts \cite{wang2015new}. 
