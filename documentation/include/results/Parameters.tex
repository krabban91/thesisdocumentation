% !TEX root = ..\..\main.tex
\section{Parameter benchmarks}
\label{sec:res:param}

As mentioned in section \ref{sec:method:eval:param} there are some parameters to evaluate in order to find the optimal setting for the proposed model. The evaluations that are presented in this section are:
\begin{itemize}
\item \textbf{\nameref{sec:res:learning}}: Described in Section \ref{sec:method:eval:param:learning} and presented in Section \ref{sec:res:learning}.
\item \textbf{\nameref{sec:res:stopping}}: Described in Section \ref{sec:method:eval:param:stopping} and presented in Section \ref{sec:res:stopping}.
\item \textbf{ \nameref{sec:res:features}}: Described in Section \ref{sec:method:eval:param:features} and presented in Section \ref{sec:res:features}.
\item \textbf{\nameref{sec:res:knownset}}: Described in Section \ref{sec:method:eval:param:training} and presented in Section \ref{sec:res:knownset}.
\end{itemize}
\medskip

\todo{Flip order: The different... in Section ref.}In Section \ref{sec:method:eval:param} the different metrics used in this evaluations are mentioned. 
However, the decision to omit some of the graphs from this thesis was made because of inconclusive differences in some metrics and an intention to only present what is relevant\todo{of relevance} in each evaluation. In most cases the information that is excluded by removing some figures from the thesis can be found by interpreting the results inbetween evaluations. But to do so should not be necessary.

Recall that in Section \ref{sec:meth:eval:bench:dataset} the datasets used for the evaluation is a subset of 23 categories drawn from the set \todo{dataset Places205}MIT Places205 (see Section \ref{sec:theory:dataset:places}) and all benchmark evaluations are performed \todo{five times with three different categories}three times with different categories as the target.

% !TEX root = ..\..\..\main.tex

\subsection{Classifier learning method}
\label{sec:res:learning}

The entirety of this evaluation is introduced in Section \ref{sec:method:eval:param:learning} and the results of the different settings are presented here. 
The four different settings of the evaluation are to present the \textbf{Top20+Bottom5}, \textbf{Top25}, \textbf{Top20+Middle5} and \textbf{Top5+Bottom20}.

As mentioned in Section \ref{sec:method:eval:param} the performance of the model is measured in two ways. The performance on an evaluation set as well the performance over the entire search space.

\subsubsection{Evaluation set}
\label{sec:res:learning:eval}
When classifying the evaluation set each iteration the measurements of the settings ended up to be very similar. The performance of the first three settings were almost identical. An initial performance peak in classifying the set that later on dropped off. While the performance of the fourth setting deviated from the performance of the others the measurements were about the same throughout all the iterations.

Independently of how images were selected the recall (see Figure \ref{fig:learning:eval_set:recall}) of the settings only differed marginally. With an exception of the fourth setting that in some manner deviated from the other ones. 
The precision of the fourth setting deviated from the other settings, which can be seen in Figure \ref{fig:learning:eval_set:precision}. Due to the low precision of the fourth setting, the F1-measure is low as well, which can be seen in Figure \ref{fig:learning:eval_set:f1}. The cause of this is that the fourth setting had an higher number of false positives than the other settings had.

An important observation to make is that towards the end of the evaluations all four settings have the same training data and therefore classifies the evaluation set accordingly. Having the complete training set results in an F1-measure around 0.32 when retrieving the category Bar and around 0.65 when retrieving the other two from the remaining 22 categories in the evaluation set.
\tripfigurenear
{include/graphs/learning/baseball_field/eval_set/precision_extra.png}
{include/graphs/learning/bedroom/eval_set/precision_extra.png}
{include/graphs/learning/bar/eval_set/precision_extra.png}
{The precison that the different settings had on the evaluation set for the three different category searches. Note that the setting Top5+Bottom20 deviates from the other settings.}
{fig:learning:eval_set:precision}

\tripfigurenear
{include/graphs/learning/baseball_field/eval_set/recall_extra.PNG}
{include/graphs/learning/bedroom/eval_set/recall_extra.PNG}
{include/graphs/learning/bar/eval_set/recall_extra.PNG}
{The recall rate on the the evaluation set.}
{fig:learning:eval_set:recall}

\tripfigurenear
{include/graphs/learning/baseball_field/eval_set/f1_extra.PNG}
{include/graphs/learning/bedroom/eval_set/f1_extra.PNG}
{include/graphs/learning/bar/eval_set/f1_extra.PNG}
{The harmonic mean of recall and precision, F1-measure, read on the evaluation set over iterations. The performance of the first three settings peak early and then drops towards the end of the evaluation.}
{fig:learning:eval_set:f1}

The accuracy when using the different settings did not vary that much either. As seen in Figure \ref{fig:learning:eval_set:accuracy} the accuracy of the settings is around 85\% when classifying the category Bar and around 95\% on the other two category evaluations. Just as with the F1-measure the accuracy measurements of the first three settings were higher in the first couple of iterations and then dropped off towards the end of the evaluation. 

\tripfigure
{include/graphs/learning/baseball_field/eval_set/accuracy_extra.PNG}
{include/graphs/learning/bedroom/eval_set/accuracy_extra.PNG}
{include/graphs/learning/bar/eval_set/accuracy_extra.PNG}
{The accuracy of the different settings on the three evaluation sets. Note how the accuracy of fourth setting varied more in-between evaluations than the accuracy of the other three settings did.}
{fig:learning:eval_set:accuracy}

\subsubsection{Search space}
\label{sec:res:learning:iter}
The different settings deviated a lot more from each other when comparing how the search space was classified compared to when the evaluation set was. The first three settings stopped predicting images as relevant early on in comparison with the fourth setting, resulting in the precision presented in Figure \ref{fig:learning:iteration:precision}. At the end of the evaluations the fourth setting had a precision of 15-25\% in all three categories while the other three settings could maintain a precision of about 80\% on the categories Bedroom and Baseball field. But when classifying the third category, Bar, all the settings had a rather low precision. In this category, the first setting was the one showing the lowest precision. While the fourth setting had a relatively low precision, it did produce the best recall over the different search spaces as seen in Figure \ref{fig:learning:iteration:recall}. But when searching for the third category it took until the final iterations before the last relevant images were retrieved. To compare this with the other settings, where the last relevant image was retrieved slightly after half of the evaluation. 

\tripfigurenear
{include/graphs/learning/baseball_field/iteration/precision_extra.PNG}
{include/graphs/learning/bedroom/iteration/precision_extra.PNG}
{include/graphs/learning/bar/iteration/precision_extra.PNG}
{The precision that the different settings had classifying the search spaces for the three evaluation categories. Note how the setting Top5+Bottom20 continued to predict images as positives when they indeed were negatives throughout the entire search.}
{fig:learning:iteration:precision}

\tripfigurenear
{include/graphs/learning/baseball_field/iteration/recall_extra.PNG}
{include/graphs/learning/bedroom/iteration/recall_extra.PNG}
{include/graphs/learning/bar/iteration/recall_extra.PNG}
{The recall rate on the three evaluation categories. The recall is slightly higher for those settings that continually present some of the bottom images in each iteration.}
{fig:learning:iteration:recall}

In terms of performance as an image retrieval system the setting of presenting a majority of negatives is not to prefer. As seen in Figure \ref{fig:learning:iteration:retrieval}, the number of retrieved images by the fourth setting is a lot lower than the number for the other settings over the entire run. Looking at the performance on the category Bar there are some iterations where the fourth setting has retrieved fewer relevant images than when selecting images at random. This is only momentarily and the rate soon returns to being slightly above that. Reading into the F1-measure in Figure \ref{fig:learning:iteration:f1} and the accuracy in Figure \ref{fig:learning:iteration:accuracy} the data is out of favor of the fourth setting. But when inspecting the readings of the category Bar the setting Top20+Bottom5 has a slightly worse F1-measure and accuracy than the three other settings. 

\tripfigurenear
{include/graphs/learning/baseball_field/iteration/retrieved_extra.PNG}
{include/graphs/learning/bedroom/iteration/retrieved_extra.PNG}
{include/graphs/learning/bar/iteration/retrieved_extra.PNG}
{The number of retrieved images over iteration that the different settings had while classifying the search spaces for the three evaluation categories. Note how the category Bar is more difficult than the other ones to retrieve relevant images from.}
{fig:learning:iteration:retrieval}
\tripfigurenear
{include/graphs/learning/baseball_field/iteration/f1_extra.PNG}
{include/graphs/learning/bedroom/iteration/f1_extra.PNG}
{include/graphs/learning/bar/iteration/f1_extra.PNG}
{The F1-measure that the different settings had classifying the search spaces for the three evaluation categories. An harmonic mean of the precision and the recall.}
{fig:learning:iteration:f1}


The setting that will be used to select the set of images each iteration needs to be picked in order to continue the parameter evaluation. When working with image retrieval the model needs to have a high retrieval rate of relevant images early on which causes the Top5+Bottom20 setting to not meet the preferences. To select between the remaining three settings one can recall what was mentioned in Chapter \ref{chapter:intro}: Investigation material needs to be retrieved in a quick manner and labeled correctly. In other words as few false negatives as possible. In all three categories; the Top20+Bottom5 setting had a higher recall rate on the search space throughout the evaluations (see Figure \ref{fig:learning:iteration:recall}). Which means that the first setting is the most appropriate one to use in the following benchmarks. 
\tripfigure
{include/graphs/learning/baseball_field/iteration/accuracy_extra.PNG}
{include/graphs/learning/bedroom/iteration/accuracy_extra.PNG}
{include/graphs/learning/bar/iteration/accuracy_extra.PNG}
{The accuracy that the different settings had classifying the search spaces for the three evaluation categories.}
{fig:learning:iteration:accuracy}

% !TEX root = ..\..\..\main.tex

\subsection{Limiting search space}
\label{sec:res:stopping}

This parameter benchmark is described in Section \ref{sec:method:eval:param:stopping} and evaluates the four settings 
\textbf{All images}, \textbf{Threshold}, \textbf{Early stopping} and \textbf{Both rules}.

The metrics used in this evaluation is presented in Section \ref{sec:method:eval:param}. Recall that the effectiveness of the model during search iterations is also measured in terms of time taken and how many images that are processed each iteration in order to find the best set of images to present to the user. Also that the performance of the model is measured on an evaluation set as well as over the entire search space. 

\subsubsection{Evaluation set}
When measuring the performance on the evaluation set, none of the stopping conditions affected the learning rate in any direction. The evaluation set was  classified in a similar manner by all the settings throughout the whole benchmark. 
Resulting in that all settings have about the same values on all metrics. Which is for instance visible when inspecting the F1-measure in Figure \ref{fig:learning:eval_set:f1}. Since all settings followed the same pattern the presentation of recall rate, precision and accuracy is omitted in this section. However the general trend of the measurements can be observed in what is referred to as \emph{Top20+Bottom5} in Section \ref{sec:res:learning:eval}.

\tripfigure
{include/graphs/stopping/baseball_field/eval_set/f1.PNG}
{include/graphs/stopping/bedroom/eval_set/f1.PNG}
{include/graphs/stopping/bar/eval_set/f1.PNG}
{The harmonic mean of recall and precision, F1-measure, read on the evaluation set over iterations. All of the settings had performed similarly throughout the evaluation.}
{fig:stopping:eval_set:f1}

Since the concept was learned, independent of which of these stopping conditions that were used, the classification performance on the search space becomes more important as well as how much time that is spent calculating distances to data points each search iteration.

\subsubsection{Search space}
\label{sec:result:stopping:iterations}
During the search iterations, the condition \emph{Early stopping} introduced a notable trend compared to the condition \emph{Threshold} in terms of correctness. As seen in Figure \ref{fig:stopping:iteration:precision}, the precision, when prediction the search space, using the stopping condition \emph{Early stopping} gradually decreased, causing the precision of using both rules to decrease as well. Since the selection of images near the decision boundary on the negative side is delayed when using the early stopping condition, the material near the decision boundary is predicted correctly due to more knowledge. Thus causing the recall to stay higher than the other settings, as seen in Figure \ref{fig:stopping:iteration:recall}.

\tripfigurenear
{include/graphs/stopping/baseball_field/iteration/precision.PNG}
{include/graphs/stopping/bedroom/iteration/precision.PNG}
{include/graphs/stopping/bar/iteration/precision.PNG}
{The precision of the different settings when classifying the search spaces of the three evaluation categories. The early stopping condition appear to have a negative impact on precision.}
{fig:stopping:iteration:precision}

\tripfigurenear
{include/graphs/stopping/baseball_field/iteration/recall.PNG}
{include/graphs/stopping/bedroom/iteration/recall.PNG}
{include/graphs/stopping/bar/iteration/recall.PNG}
{The recall rate of the different settings classifying the search spaces of the three evaluation categories. The condition \emph{Early stopping} may have a positive effect on recall.}
{fig:stopping:iteration:recall}

If no stopping condition is used the total number of handled images, when the total search space is initially 4600 images, is \ref{eq:res:stopping:images}

\begin{equation}
\label{eq:res:stopping:images}
\textnormal{Handled images}_{max} = \sum_{i=1}^{4600/25}{25i} =\sum_{i=1}^{184}{25i} = 425500.
\end{equation}

The total number of handled images for the different stopping conditions settings in relation to this number can be seen in Table \ref{table:res:stopping:images}. This indicates that the reduction of total handled images when combining the stopping conditions is $\approx80-85\%$ on this dataset. Yet the F1-measure levels were only reduced by 20-25\% according to the result in Figure \ref{fig:stopping:iteration:f1}. 

\begin{table}
\centering
\begin{tabular}{l | p{2.6cm} p{2.6cm} p{3.2cm} p{1.6cm} }
\textbf{Set} & \textbf{All images} & \textbf{Threshold} & \textbf{Early stopping} & \textbf{Both} \\\hline
Bedroom & 1 & 0.4168 & 0.2918 & 0.1522 \\
Bar & 1 & 0.2537 & 0.5717 & 0.1862 \\
Baseball field & 1 & 0.3843 & 0.2770 & 0.1491 
\end{tabular}
\caption{The average of total images that are handled during a search for each setting. The ratios of the value in Equation \ref{eq:res:stopping:images} are presented. The values are the average of five evaluations for each setting.}
\label{table:res:stopping:images}
\end{table}

\tripfigurenear
{include/graphs/stopping/baseball_field/iteration/f1.PNG}
{include/graphs/stopping/bedroom/iteration/f1.PNG}
{include/graphs/stopping/bar/iteration/f1.PNG}
{The F1-measure that the different settings had when classifying the search spaces of the three evaluation categories.}
{fig:stopping:iteration:f1}

When inspecting the number of handled images by the different settings per iteration in Figure \ref{fig:stopping:iteration:handled} it becomes clear how the reduction of handled images could vary so much between the classes when using each condition solely. In the evaluation of retrieving the category \emph{Bar} the model finds more images that are predicted as relevant in a more dense manner. Which is why the early stopping condition does not evaluate as true for most of the early iterations and why the threshold setting stops exploring earlier every iteration than in the other two evaluation categories. The two stopping conditions are so disjunct in terms of when their conditions are met that the combination of the two actually reduce the number of handled images in the search for all three categories. 

\tripfigurenear
{include/graphs/stopping/baseball_field/iteration/handled.PNG}
{include/graphs/stopping/bedroom/iteration/handled.PNG}
{include/graphs/stopping/bar/iteration/handled.PNG}
{The number of handled images every iteration for the different settings. Having both stopping conditions results in processing fewer images in all three categories.}
{fig:stopping:iteration:handled}


In terms of time taken there is a notable improvement when using a stopping condition instead of handling all the images in the search space. The time that an iteration takes varies a lot for the condition \emph{Early stopping}, see Figure \ref{fig:stopping:iteration:timetaken}, while the time taken for the threshold condition is just about the same throughout the entire search. But depending on which category that the algorithm is looking for the total time spent searching for the two conditions seems to vary compared to each other and combining the two conditions results in less time spent in all evaluations according to the measurements in Figure \ref{fig:stopping:iteration:totaltime}. On average, the total time reduction when using both stopping conditions compared with classifying all images every iteration was $\approx60\%$. 

\tripfigurenear
{include/graphs/stopping/baseball_field/iteration/timetaken.PNG}
{include/graphs/stopping/bedroom/iteration/timetaken.PNG}
{include/graphs/stopping/bar/iteration/timetaken.PNG}
{The time taken every iteration for the different settings. Having both stopping conditions results in the lowest peak in terms of time taken in all categories.}
{fig:stopping:iteration:timetaken}

\tripfigurenear
{include/graphs/stopping/baseball_field/iteration/totaltime.PNG}
{include/graphs/stopping/bedroom/iteration/totaltime.PNG}
{include/graphs/stopping/bar/iteration/totaltime.PNG}
{The total time taken for the different settings. There is a slight correlation with the ratios of processed images presented in Table \ref{table:res:stopping:images} and the time taken ratios in these graphs.}
{fig:stopping:iteration:totaltime}

In the same line as the F1-measure results, the number of retrieved images was a bit lower for the condition \emph{Early stopping} (see Figure \ref{fig:stopping:iteration:retrieval}). 
When searching for the category \emph{Baseball field} the early stopping condition seem to delay the full retrieval of the category by about 80 iterations. But when retrieving material from the category \emph{Bar} the third setting follows the trend of the other ones and retrieves the relevant images at the same pace. Which could be caused by the same reason that was observed in Figure \ref{fig:stopping:iteration:handled}; the \emph{Early stopping} setting often samples material until no more unique images are found. It is however clear that the setting \emph{Both} follows the setting \emph{Early stopping} in terms of how many relevant images that are retrieved at a certain iteration. 


\tripfigure
{include/graphs/stopping/baseball_field/iteration/retrieved.png}
{include/graphs/stopping/bedroom/iteration/retrieved.png}
{include/graphs/stopping/bar/iteration/retrieved.png}
{The number of retrieved relevant images over iterations for the different settings when classifying the three evaluation categories. The threshold setting retrieves images in the same rate as processing the entire search space each iteration. On the category Baseball field, the early stopping setting requires about 140 iterations to retrieve all 200 images.}
{fig:stopping:iteration:retrieval}

% !TEX root = ..\..\..\main.tex

\subsection{Feature descriptors}
\label{sec:res:features}
This evaluation is described in Section \ref{sec:method:eval:param:features} and a presentation of the evaluated settings can be found there. The settings described are \textbf{HOG}, \textbf{GCH}, \textbf{WT}, \textbf{CNN}, \textbf{Edge}, \textbf{All} and \textbf{All-CNN}.

The evaluation is performed with both of the stopping conditions that were evaluated in Section \ref{sec:res:stopping} and the material that is presented in each iteration is, as the evaluation in Section \ref{sec:res:learning:iter} suggests, a combination of the top-20 images and the bottom-5. 

Just like the previous parameter benchmarks the performance is measured over the entire search space as well as how a separate evaluation set is categorized.

\subsubsection{Evaluation set}
\label{sec:res:features:eval}


Throughout all the metrics that were measured during the evaluation the settings that used CNN feature descriptors considerably outperformed the other settings.  A trend that is distinguishable in both the F1-measure (Figure \ref{fig:features:eval_set:f1}) and the accuracy (Figure \ref{fig:features:eval_set:accuracy}) of the two settings that incorporate the CNN feature descriptor. There was a slight difference between using only the CNN feature vector as a descriptor and combining all the feature descriptors.

\tripfigurenear
{include/graphs/features/baseball_field/eval_set/f1.PNG}
{include/graphs/features/bedroom/eval_set/f1.PNG}
{include/graphs/features/bar/eval_set/f1.PNG}
{The F1-measure read on the evaluation set over iterations. Combining the information in different feature descriptors seems to give equally good or better results compared to the best single descriptor.}
{fig:features:eval_set:f1}

\tripfigurenear
{include/graphs/features/baseball_field/eval_set/accuracy.PNG}
{include/graphs/features/bedroom/eval_set/accuracy.PNG}
{include/graphs/features/bar/eval_set/accuracy.PNG}
{The accuracy of the different settings on the three evaluation sets. The CNN feature descriptor performs better as an information vector than the other descriptors.}
{fig:features:eval_set:accuracy}

The value of combining different feature descriptors become more clear when omitting the results of the fourth and sixth setting and only inspecting the remaining four feature descriptors and the combination of those. The F1-measure of the remaining five settings can be seen in Figure \ref{fig:features:eval_set:f1_no_cnn}. Which single feature descriptor that performs the best depends on which category that the target is. For the category Bedroom a better result is given when using the HOG descriptor while for the other two categories the target concept is more distinguishable when using the GCH descriptor. But more importantly; the combination of the four descriptor performs better or just as good as the best single descriptor. 


\tripfigure
{include/graphs/features/baseball_field/eval_set/f1_no_cnn.PNG}
{include/graphs/features/bedroom/eval_set/f1_no_cnn.PNG}
{include/graphs/features/bar/eval_set/f1_no_cnn.png}
{A closer look on the F1-measure of all the settings that are not handling feature vectors derived from a neural network. The combination of feature descriptors results in equally good or better results compared to the best single descriptor.}
{fig:features:eval_set:f1_no_cnn}

\subsubsection{Search space}
\label{sec:res:features:iter}

As mentioned in Section \ref{sec:res:features:eval} the different target categories of the data sets are more distinguishable when using the CNN activation vector during classification. This is also evident in the F1-measure (Figure \ref{fig:features:iteration:f1}) and accuracy (Figure \ref{fig:features:iteration:accuracy}) when classifying the search space. More interestingly the positive effect of combining the first order classifiers become more clear in these measurements. When inspecting the  category Bar in the two figures the setting All outperforms the setting that only uses the CNN activation vector as a feature descriptor.

\tripfigurenear
{include/graphs/features/baseball_field/iteration/f1.PNG}
{include/graphs/features/bedroom/iteration/f1.PNG}
{include/graphs/features/bar/iteration/f1.PNG}
{The F1-measure that the different settings had when classifying the search spaces of the three evaluation categories. The F1-measure on the category Bar is higher when combining all the feature descriptors compared with solely using the CNN feature descriptor.}
{fig:features:iteration:f1}

\tripfigurenear
{include/graphs/features/baseball_field/iteration/accuracy.PNG}
{include/graphs/features/bedroom/iteration/accuracy.PNG}
{include/graphs/features/bar/iteration/accuracy.PNG}
{The accuracy of the different settings when classifying the search spaces of the three evaluation categories. As in Figure \ref{fig:features:iteration:f1}, there seems to be have a positive effect of combining feature descriptors.}
{fig:features:iteration:accuracy}


All the different settings of feature descriptors do seem to be able to present some distinguishability between the different categories. With the exception of the initial rounds of searching for the category Baseball field with the setting WT, where the results are in line with selecting images at random. The number of retrieved images at a certain iteration is presented in Figure \ref{fig:features:iteration:retrieval}. Here the superiority of using the CNN activation vector becomes even more clear: When retrieving the bedroom class, the settings that use the CNN feature descriptor have retrieved the entire target category at iteration 80 while the other settings can not retrieve the entire class until the search space is depleted. 

\tripfigurenear
{include/graphs/features/baseball_field/iteration/retrieved.png}
{include/graphs/features/bedroom/iteration/retrieved.png}
{include/graphs/features/bar/iteration/retrieved.png}
{The number of retrieved images that the different settings had when classifying the search spaces. By introducing the CNN feature descriptors to the classifier the concepts of the categories seem to be easier to retrieve.}
{fig:features:iteration:retrieval}


The results of the setting that uses all the feature descriptors are considerably high in terms of pace of retrieving images as well as with the F1-measure and the accuracy. But combining feature descriptors does come with the drawback of fitting more classifiers each iteration, causing the total time taken to grow accordingly.
The total time spent computing predictions depended on how many and which classifiers that were used and by using all six classifiers the time taken grew accordingly. In Figure \ref{fig:features:iteration:totaltime} it is notable that combining all five descriptors compared to only combining four descriptors causes the computation time to grow to almost the double. 

\tripfigure
{include/graphs/features/baseball_field/iteration/totaltime.PNG}
{include/graphs/features/bedroom/iteration/totaltime.PNG}
{include/graphs/features/bar/iteration/totaltime.PNG}
{The total time taken for the different settings. The more feature descriptors that are used the more time it takes to train the classifying system.}
{fig:features:iteration:totaltime}

Given the performance of the CNN setting one might consider only using this setting in the final evaluation. But as stated in Section \ref{sec:method:eval:param} the classifier is used as intended throughout the parameter benchmarks with the exception of this evaluation. 


% !TEX root = ..\..\..\main.tex

\subsection{Training data}
\label{sec:res:knownset}

The results of the evaluation described in Section \ref{sec:method:eval:param:training} are presented here. The settings of the evaluation are listed in Table \ref{table:res:param:training:settings}. 

\begin{table}[H]
\centering
\begin{scriptsize}

\begin{tabular}{L{2cm} | L{4cm} | L{5cm}}
\textbf{Setting} & \textbf{Training} & \textbf{Predefined training set \emph{(relevant+irrelevant)}}\\\hline
1A & Only the first iteration & 5+5 \\
1B & Only the first iteration & 5+50  \\
1C & Only the first iteration & 22+484 \\
1D & Only the first iteration & 250+250 \\
2A & Every iteration & 5+5 \\
2B & Every iteration & 5+50 \\
2C & Every iteration & 22+484 \\
2D & Every iteration & 250+250 \\
3 & Every iteration & 0+0 \\
\end{tabular}
\end{scriptsize}
\caption{The different settings evaluated in this benchmark. Deeper explanation found in Section \ref{sec:method:eval:param:training}.}
\label{table:res:param:training:settings}
\end{table}

The metrics used for this evaluation are presented in Section \ref{sec:method:eval:param} and are as mentioned performed on an evaluation set and the search space. But the number of settings in this evaluation was higher than anticipated. Due to the number of different settings in this evaluation the presentation of the metrics will only consist the average value of the five runs of each setting. When the graphs included minimum and maximum values for each setting the data became much harder to understand and close to impossible to draw any conclusions from.

The model will during the evaluation use all of the stopping conditions described in Section \ref{sec:method:proposed:matching:search}, present the \emph{top-20} images and \emph{bottom-5} images in the end of every iteration as proposed in Section \ref{sec:result:stopping:iterations} and the classifier uses the full set of feature descriptors as described in Section \ref{sec:method:eval:param}. 

\subsubsection{Evaluation set}
\label{sec:res:knownset:eval}
Naturally the performance of the settings that only use a predefined training set when classifying the evaluation set is same throughout all iterations. As expected the performance increases when having more data to begin with. In terms of F1-measure (seen in Figure \ref{fig:knownset:evalset:f1}) the performance of smaller training data sets are superseded by the performances of bigger training data sets. 


The settings that use training set data retrieved by relevance feedback did however outperform the settings that did not. In the end of the search both the F1-measure and the accuracy where about the same for setting \emph{3} as for the setting \emph{1D}. But at around iteration 25 in the settings that use relevance feedback (settings \emph{2A-2D} and \emph{3}) have performance peaks that are considerably higher than at the end of the search, which is noticeable in Figure \ref{fig:knownset:evalset:f1} and Figure \ref{fig:knownset:evalset:accuracy}.

\tripfigurenear
{include/graphs/knownset/baseball_field/eval_set/f1.PNG}
{include/graphs/knownset/bedroom/eval_set/f1.PNG}
{include/graphs/knownset/bar/eval_set/f1.PNG}
{The F1-measure read on the evaluation set over iterations.}
{fig:knownset:evalset:f1}

The biggest difference between the settings \emph{2A-2D} and the setting \emph{3} was that during the initial three to six iterations. The setting with predefined data could actually perform a search while setting \emph{3} could not. Causing the performance of setting \emph{3} to be considerably lower than the performance of the other settings. But after those iterations the performance continuously rose up to be in level with the the settings \emph{2A, 2B, 2C \& 2D}. Eventhough setting \emph{2D} had a predefined data set of 250+250 relevant and non-relevant images their performance on the evaluation set turned out to be level. Yet it took the setting \emph{2D} a few more iterations to be on par with the settings \emph{2A-2C} and \emph{3}.

\tripfigure
{include/graphs/knownset/baseball_field/eval_set/accuracy.PNG}
{include/graphs/knownset/bedroom/eval_set/accuracy.PNG}
{include/graphs/knownset/bar/eval_set/accuracy.PNG}
{The accuracy on the evaluation sets. Using only training set data from relevance (setting \emph{3}) achieves higher results than a predefined training set, that has more relevant images than the search space (setting \emph{1D}) between iteration 10 and 80.}
{fig:knownset:evalset:accuracy}

\subsubsection{Search space}
\label{sec:res:knownset:iter}

The results of measuring how the different settings classified the search space were not as indicating as the results that were observed when classifying the evaluation set. 
The metrics precision, recall, F1-measure and accuracy gave relatively inconclusive results. The values of the F1-measure of the evaluation (seen in Figure \ref{fig:knownset:iteration:f1}) do however imply some trends. In the first couple of iterations of the search the settings that use traning set data extracted from relevance feedback begin to improve their results while the other settings begins with a high performance that shortly after decreases and then stabilizes.  

\tripfigurenear
{include/graphs/knownset/baseball_field/iteration/f1.PNG}
{include/graphs/knownset/bedroom/iteration/f1.PNG}
{include/graphs/knownset/bar/iteration/f1.PNG}
{The F1-measure that the different settings had when classifying the search space. Having more data implies that the results become better.}
{fig:knownset:iteration:f1}

In terms of how well the different settings perform these measurements do not imply anything else than the more training data one has the better the result gets. The number of retrieved images after a certain iteration, as visualized in Figure \ref{fig:knownset:iteration:retrieval}, reflects the different sizes of traning data they had. By comparing the results of setting \emph{1D} with setting \emph{2D} when searching for the category \emph{Bar} in this figure, one can see how the data from relevance feedback improved the result of retrieving relevant material.

\tripfigurenear
{include/graphs/knownset/baseball_field/iteration/retrieved.png}
{include/graphs/knownset/bedroom/iteration/retrieved.png}
{include/graphs/knownset/bar/iteration/retrieved.png}
{The number of retrieved imaged over the search iterations. Solely having 5 relevant images as training data (as in setting \emph{1A} and setting \emph{1B}) results in the incapability of retrieving the full content of the search space until it is depleted.}
{fig:knownset:iteration:retrieval}

In the intended scenario, the search space is unlabeled and is therefore categorized while the data is presented. In this setting the time taken to predict the category of the images should be relative to how long it takes to correct poorly predicted categories, which is not covered by the scope of the thesis. The time taken for each of the settings to categorize the search space relative to each other is presented in Figure \ref{fig:knownset:iteration:timetaken}. The impact of refitting the classifiers each iteration truly becomes clear in this figure as well as how the size of the training set causes training the classifier to take longer. 

\tripfigure
{include/graphs/knownset/baseball_field/iteration/totaltime.PNG}
{include/graphs/knownset/bedroom/iteration/totaltime.PNG}
{include/graphs/knownset/bar/iteration/totaltime.PNG}
{The total time taken for the different settings of this evaluation. There is a huge time difference between the training the classifier once and doing it every iteration.}
{fig:knownset:iteration:timetaken}

