% CREATED BY DAVID FRISK, 2016
% !TEX root = ..\..\main.tex

% BASIC SETTINGS
\usepackage{moreverb}								% List settings
\usepackage{textcomp}								% Fonts, symbols etc.
\usepackage{lmodern}								% Latin modern font
\usepackage{helvet}									% Enables font switching
\usepackage[T1]{fontenc}							% Output settings
\usepackage[english]{babel}							% Language settings
\usepackage[utf8]{inputenc}							% Input settings
\usepackage{amsmath}								% Mathematical expressions (American mathematical society)
\usepackage{amssymb}								% Mathematical symbols (American mathematical society)
\usepackage{graphicx}								% Figures
\usepackage{subfig}									% Enables subfigures
\numberwithin{equation}{chapter}					% Numbering order for equations
\numberwithin{figure}{chapter}						% Numbering order for figures
\numberwithin{table}{chapter}						% Numbering order for tables
\usepackage{listings}								% Enables source code listings
\usepackage{chemfig}								% Chemical structures
\usepackage[top=3cm, bottom=3cm,
			inner=3cm, outer=3cm]{geometry}			% Page margin lengths			
\usepackage{eso-pic}								% Create cover page background
\newcommand{\backgroundpic}[3]{
	\put(#1,#2){
	\parbox[b][\paperheight]{\paperwidth}{
	\centering
	\includegraphics[width=\paperwidth,height=\paperheight,keepaspectratio]{#3}}}}
\usepackage{float} 									% Enables object position enforcement using [H]
\usepackage{parskip}								% Enables vertical spaces correctly 
% PACKAGE USAGES


% OPTIONAL SETTINGS (DELETE OR COMMENT TO SUPRESS)

% Lorem ipsum package
\usepackage[toc,page]{appendix}

\usepackage{etoolbox}





% Disable automatic indentation (equal to using \noindent)
\setlength{\parindent}{0cm}                         


% Caption settings (aligned left with bold name)
\usepackage[labelfont=bf, textfont=normal,
			justification=justified,
			singlelinecheck=false]{caption} 		

		  	
% Activate clickable links in table of contents  	
\usepackage{hyperref}								
\hypersetup{colorlinks, citecolor=black,
   		 	filecolor=black, linkcolor=black,
    		urlcolor=black}


% Define the number of section levels to be included in the t.o.c. and numbered	(3 is default)	
\usepackage[subfigure, titles]{tocloft}

\setlength{\cftbeforechapskip}{6pt}
\setcounter{tocdepth}{5}							
\setcounter{secnumdepth}{5}	


% Chapter title settings
\usepackage{titlesec}		
\titleformat{\chapter}[display]
  {\Huge\bfseries\filcenter}
  {{\fontsize{50pt}{1em}\vspace{-6.8ex}\selectfont \textnormal{\thechapter}}}{0ex}{}[]

  \titlespacing{\chapter}{0pt}{3em}{1em}





% Header and footer settings (Select TWOSIDE or ONESIDE layout below)
\usepackage{fancyhdr}								
\pagestyle{fancy}  
\renewcommand{\chaptermark}[1]{\markboth{\thechapter.\space#1}{}} 


% Select one-sided (1) or two-sided (2) page numbering
\def\layout{2}	% Choose 1 for one-sided or 2 for two-sided layout
% Conditional expression based on the layout choice
\ifnum\layout=2	% Two-sided
    \fancyhf{}			 						
	\fancyhead[LE,RO]{\nouppercase{ \leftmark}}
	\fancyfoot[LE,RO]{\thepage}
	\fancypagestyle{plain}{			% Redefine the plain page style
	\fancyhf{}
	\renewcommand{\headrulewidth}{0pt} 		
	\fancyfoot[LE,RO]{\thepage}}	
\else			% One-sided  	
  	\fancyhf{}					
	\fancyhead[C]{\nouppercase{ \leftmark}}
	\fancyfoot[C]{\thepage}
\fi


% Enable To-do notes
\usepackage[textsize=tiny]{todonotes}   % Include the option "disable" to hide all notes
\setlength{\marginparwidth}{2.5cm} 


% Supress warning from Texmaker about headheight
\setlength{\headheight}{15pt}		


\usepackage{enumitem}



% vectors
\renewcommand{\vec}[1]{\mathbf{#1}}

% New commands. 
\newcommand{\TODO}[1]{
	\todo[inline]{#1}
}

\newcommand{\todog}[1]{
	\todo[color=green!40]{#1}
}
\newcommand{\todokahl}[1]{
	\todo[color=red!60]{#1}
}


\usepackage{array}
\newcolumntype{L}[1]{>{\raggedright\let\newline\\\arraybackslash\hspace{0pt}}m{#1}}

\newcommand{\tripfig}[5]{
	\begin{figure}
		\centering
		\begin{tabular}{l@{\tiny{ }} @{\small{ }}r}
			%\includegraphics[width=0.39205\textwidth]{#1} &
			%\includegraphics[width=0.39205\textwidth]{#2} 
			%\includegraphics[width=0.44205\textwidth]{#1} &
			%\includegraphics[width=0.44205\textwidth]{#2}
			\includegraphics[width=0.24205\textwidth]{#1} &
			\includegraphics[width=0.24205\textwidth]{#2}
		\end{tabular}
		\centering
	  	\begin{tabular}{@{}c@{}}
			%\includegraphics[width=0.8\textwidth]{#3} 
			%\includegraphics[width=0.9\textwidth]{#3} 
			%\includegraphics[width=0.6\textwidth]{#3} 
			\includegraphics[width=0.5\textwidth]{#3} 
		\end{tabular}
		\caption{#4}
		\label{#5}
	\end{figure}
}

\newcommand{\quadfigure}[6]{
	\begin{figure}
		\centering
		\begin{tabular}{l@{\tiny{ }} @{\small{ }}r}
			%\includegraphics[width=0.39205\textwidth]{#1} &
			%\includegraphics[width=0.39205\textwidth]{#2} 
			%\includegraphics[width=0.29205\textwidth]{#1} &
			%\includegraphics[width=0.29205\textwidth]{#2}
			\includegraphics[width=0.27305\textwidth]{#1} &
			\includegraphics[width=0.27305\textwidth]{#2}
			%\includegraphics[width=0.44205\textwidth]{#1} &
			%\includegraphics[width=0.44205\textwidth]{#2}
		\end{tabular}
		\centering
	  	\begin{tabular}{l@{\tiny{ }} @{\small{ }}r}
			%\includegraphics[width=0.39205\textwidth]{#3} &
			%\includegraphics[width=0.39205\textwidth]{#4} 
			%\includegraphics[width=0.29205\textwidth]{#3} &
			%\includegraphics[width=0.29205\textwidth]{#4}
			\includegraphics[width=0.27305\textwidth]{#3} &
			\includegraphics[width=0.27305\textwidth]{#4}
			%\includegraphics[width=0.44205\textwidth]{#3} &
			%\includegraphics[width=0.44205\textwidth]{#4}
		\end{tabular}
		\caption{#5}
		\label{#6}
	\end{figure}
}

\newcommand{\singlefigure}[4]{
	\begin{figure}[H]
		\centering
		\includegraphics[width=#4\textwidth]{#1}
		\caption{#2}
		\label{#3}
	\end{figure}
}
\newcommand{\singlefigurenear}[4]{
	\begin{figure}
		\centering
		\includegraphics[width=#4\textwidth]{#1}
		\caption{#2}
		\label{#3}
	\end{figure}
}


\newcommand{\tripfigure}[5]{
	\begin{figure}[H]
		\centering
		\begin{tabular}{l@{\tiny{ }} @{\small{ }}r}
			\begin{tabular}{@{}c@{}}
				%\includegraphics[width=0.5\textwidth]{#3}
				%\includegraphics[width=0.8\textwidth]{#3} 
				%\includegraphics[width=0.9\textwidth]{#3} 
				\includegraphics[width=0.6\textwidth]{#3} 
				%\includegraphics[width=0.55\textwidth]{#3} 
			\end{tabular} &
		  	\begin{tabular}{@{}c@{}}
				%\includegraphics[width=0.24205\textwidth]{#1} \\
				%\includegraphics[width=0.24205\textwidth]{#2}
				%\includegraphics[width=0.39205\textwidth]{#1} \\
				%\includegraphics[width=0.39205\textwidth]{#2} 
				%\includegraphics[width=0.44205\textwidth]{#1} \\
				%\includegraphics[width=0.44205\textwidth]{#2}
				\includegraphics[width=0.29205\textwidth]{#1} \\
				\includegraphics[width=0.29205\textwidth]{#2}
				%\includegraphics[width=0.27305\textwidth]{#1} \\
				%\includegraphics[width=0.27305\textwidth]{#2}
			\end{tabular}
		\end{tabular}
		\centering
		\caption{#4}
		\label{#5}
	\end{figure}
}
\newcommand{\tripfigurenear}[5]{
	\begin{figure}
		\centering
		\begin{tabular}{l@{\tiny{ }} @{\small{ }}r}
			\begin{tabular}{@{}c@{}}
				%\includegraphics[width=0.45\textwidth]{#3}
				%\includegraphics[width=0.5\textwidth]{#3}
				%\includegraphics[width=0.8\textwidth]{#3} 
				%\includegraphics[width=0.9\textwidth]{#3} 
				\includegraphics[width=0.6\textwidth]{#3} 
			\end{tabular} &
		  	\begin{tabular}{@{}c@{}}
				%\includegraphics[width=0.21705\textwidth]{#1} \\
				%\includegraphics[width=0.21705\textwidth]{#2}
				%\includegraphics[width=0.24205\textwidth]{#1} \\
				%\includegraphics[width=0.24205\textwidth]{#2}
				%\includegraphics[width=0.39205\textwidth]{#1} \\
				%\includegraphics[width=0.39205\textwidth]{#2} 
				%\includegraphics[width=0.44205\textwidth]{#1} \\
				%\includegraphics[width=0.44205\textwidth]{#2}
				\includegraphics[width=0.29205\textwidth]{#1} \\
				\includegraphics[width=0.29205\textwidth]{#2}
			\end{tabular}
		\end{tabular}
		\centering
		\caption{#4}
		\label{#5}
	\end{figure}
}


\hyphenpenalty=100000

\newcommand{\twofigure}[4]{
	\begin{figure}[H]
		\centering
		\begin{tabular}{c @{\tiny{ }} c}
			\includegraphics[width=0.39205\textwidth]{#1} &
			\includegraphics[width=0.39205\textwidth]{#2} 
		\end{tabular}
		\caption{#3}
		\label{#4}
	\end{figure}
}

\newcommand{\fourfigure}[6]{
	\begin{figure}[H]
		\centering
		\begin{tabular}{c@{\tiny{ }} c@{\tiny{ }} c@{\tiny{ }} c}
			\includegraphics[width=0.24205\textwidth]{#1} &
			\includegraphics[width=0.24205\textwidth]{#2} &
			\includegraphics[width=0.24205\textwidth]{#3} &
			\includegraphics[width=0.24205\textwidth]{#4} 
		\end{tabular}
		\caption{#5}
		\label{#6}
	\end{figure}
}

\newcommand{\threefigure}[5]{
	\begin{figure}[H]
		\centering
		\begin{tabular}{c@{\tiny{ }} c@{\tiny{ }} c@{\tiny{ }} c}
			\includegraphics[width=0.29205\textwidth]{#1} &
			\includegraphics[width=0.29205\textwidth]{#2} &
			\includegraphics[width=0.29205\textwidth]{#3} 
		\end{tabular}
		\caption{#4}
		\label{#5}
	\end{figure}
}