% !TEX root = ..\main.tex
\chapter{Machine learning theory}
\label{chapter:mltheory}

This chapter introduces key concepts within the machine learning subfield of out thesis. It will start with a brief overview about machine learning theory and how supervised learning works. The chapter will continue with the classification in general and the basic ideas of Support vector machines. It will then conclude with the key feature of ensemble learning and Deep support vector machines. 

% \section{Machine Learning}
% \label{sec:machine_learning}

Machine Learning is a field in computer science focused on methods where the computers learn without being explicitly programmed. Thus can be said to be the study and construction of algorithms that can learn from data and make predictions based upon it. In order for a person to learn about something new, the person looks back on previously learned knowledge and machine learning algorithms do not differ from this pattern. A famous quote that describes machine learning by Tom M. Mitchell is \textit{``A computer program is said to learn from experience E with respect to some class of tasks T and performance measure P, if its performance at tasks in T, as measured by P, improves with experience E''} \cite{Mitchell:1997:ML:541177}. 
Machine learning is applied to several different subjects in a number of different fields. Even though there are several learning methods and utilities of machine learning this thesis will only handle the concepts of supervised learning and classification. Supervised learning will be used as it is a rewarding method when labeled data is attainable.

\section{Supervised learning}
Supervised learning is a training method for a computer program, where labeled data is explicitly used. Labeled data is a group of samples $\vec{x} \in \vec{X} $ composed of some form of information, or data, distinguishing the samples $\vec{x}_i $ where $i = \{1,...,t\} $ and labels $y_i \in \vec{Y}$, or targets, corresponding each to one sample. The labeled data is usually split into a training set, a validation set and a test set. The training set is the data used to train some form of \todo{Gap in sentence: algorithm, so if presented unknown data, it }algorithm, so if presented unknown data, it should be able to determine which samples belong to which label\todo{check sentance flow}. So presented with a set of samples $\vec{x_j}$ where $j = \{1,...,s\}$ it should predict the correct labels $y_j$. The validation set is then applied to get an idea of how its\todo{it is} performing and to see if further changes is needed to get an acceptable result. If one tries multiple approaches, this should determine which to use if the learning algorithm performs as anticipated. The last part of the labeled data, the test set, is then used to check what a possible expectation of the algorithm could be when presented unlabeled data. When these three\todo{what three parts?} parts are completed the supervised learning is done. 


% !TEX root = ..\..\main.tex
\section{Classification}

Classification is a general problem in pattern recognition where some form of input value should result in an output value which is representing the label of the element. The research behind classification is extensive and many different fields are working with classification. The choice of which classification method that should be used varies depending on the data. It is of note that no one classifier suits all cases and no one classifier outperforms every other in every other case as per the ``no free lunch'' theorem from Wolpert and Macready \cite{nflTheorem}. The most basic form of classification is binary classification where the data is separated into two categories, for instance as relevant and non-relevant. The input to these algorithms are more commonly known as feature vectors, $\vec{x_i}$. Since the number of dimension of these vectors might vary in-between different feature descriptors, the choice of classifier type is important. Since Support vector machines (SVMs) are good at handling feature vectors of both small and large numbers of dimensions, are fast at classifying and are relatively tolerant to noise, this type of classifier is used in this thesis \cite{kotsiantis2007supervised}\cite{kadam2016study}. 



% !TEX root = ..\..\main.tex
\section{Support vector machines}
\label{sec:mltheory:svm}

\todo{somewhere in this paragraph: Insert that it is a geometrical comparison.}A Support vector machine (SVM) is essentially a supervised learning model where data is analyzed for classification and regression analysis. Can be said to be a non-probabilistic \todo{does this sentence explain non-probabilistac-ness?}binary classifier since new examples are assigned to either of two categories\todo{rephrase}. The SVM model is a representation of the examples\todo{of the examples?} as parts in space that are mapped in a way, as clear as possible, that separates the classes by a margin. A data point of a set is considered as a p-dimensional vector (composed of p numbers) and the goal is to be able to separate the data with a (p-1)-dimensional hyperplane. 

SVMs have been found to function well on both small and large numbers\todo{size} of dimensions \cite{vert2005kernel}, but all datasets are not easily divided by a linear model.  Due to this the \emph{Kernel trick} was invented. A non-linear classification implicitly mapping their inputs into a different dimensional feature space.

SVMs can be used for classification, regression and outlier detection. But since the thesis has its focus within classification the theory covering the other two use cases will be omitted. 

In order to classify an SVM constructs a hyperplane, or a set of hyperplanes in higher dimensional spaces, that can be used to classify data points depending on which side of the hyperplane they reside. \todo{rephrase}Due to the limitation the number of sides that exist of an hyperplane this becomes a binary classification and therefore the label set becomes binarily defined as $\mathbf{Y} \in \{1,-1\}$. The optimal hyperplane $\vec{w}^T\vec{x}+b = 0$, as seen in Figure \ref{fig:svm_margin}, is found when the given training data is separated with an as large margin $\gamma = \frac{1}{||{\vec{w}||}}$ as possible. Which means that it will also be found when minimizing $||\vec{w}||$, \todo{and instead of all else}as well as when minimizing $\vec{w}^T\vec{w}$. 
Given training data points, or vectors, $\vec{x_i} \in \mathbb{R}^p, i=1...n$ and the respective label $y_i \in \mathbf{Y}$ the primal (\ref{eq:svm_primal})
\begin{equation}
\label{eq:svm_primal}
\begin{split}
\min_{\vec{w}\in\mathbb{R}^p,b\in\mathbb{R}}{\vec{w}^T\vec{w}}& \\
\textnormal{subject to} & \\
\forall j\textnormal{: } (\vec{w}^T\vec{x}_j + b)y_j & \geq 1
\end{split}
\end{equation}
can be constructed.

As soon as a stable hyperplane has been found the test set can be classified by simply checking which side, of the hyperplane, the data points end up on by computing the label value (\ref{eq:svm_test})
\begin{equation}
\label{eq:svm_test}
y_i = sign\left(\mathbf{w}^T\mathbf{x}_i+b\right).
\end{equation}
The larger the distance from the hyperplane to a point the more certain the prediction is that the data point belongs to a certain category. Hence the data points that are within then margin of the hyperplane have the most uncertain predictions. This can in fact be used in order to calculate some certainty that a data point belongs to a class or not. If the distance between two data points and the decision boundary compared is of different sizes, the point with the greater distance is more probable to be of the desired category \cite{tong2001support}\todo{rephrase}. 


\singlefigure
{figure/SVM_margins.png}
{A simplified visualization of how data is linearly separable in a two dimensional space.}
{fig:svm_margin}
{1}


\subsection{Kernels}

Kernels define the cartesian product between vectors, which can be used to get the a real value. In order to create a kernel one defines a function $K$ from the cartesian product of the feature space to a real value ($K:\mathcal{X}\times \mathcal{X} \rightarrow \mathbb{R}$). This real value can subsequently be used to evaluate a distance value. Depending on the chosen kernel the distance have different attributes an thus the kernel can be selected to receive better results for different datasets.
Kernels in SVMs are different methods of how the hyperplane is generated for the SVM and thus gives different ways of separating the data.  The most common ones are the \emph{linear} and the \emph{radial-basis function} (RBF) kernels. The linear kernel is a straightforward approach which is as the name suggests a separates data in in a linear manner. 
But when data data is not linearly separable this kernel will not suffice. 
An example of a kernel that could solve solve this problem is the RBF. The RBF kernel is a fast approach that often work, as long as the feature space is not too large, and can be used to separate inliers from outliers. The kernel often uses a norm function between two points in order to separate them, e.g. the squared two-norm  (\ref{eq:rbf_kernel})
\begin{equation}
\label{eq:rbf_kernel}
K_{RBF}(\textbf{x},\textbf{y}) = ||\textbf{x}-\textbf{y}||_2 ^2=\sum_{i=1}^p  (\textbf{x}_i-\textbf{y}_i)^2.
\end{equation}
Different kernels are used to make data points linearly separable in their own dimensional spaces, causing the decision boundary in the original feature space to be, and appear, non-linear.


% !TEX root = ..\..\main.tex

\section{Ensemble learning}

Ensemble learning methods use a setup of different learning methods which are then combined in to achieve a better predictive behavior than if using a single one. There is no guarantee that the result will be better with only combining several different\todo{wording} methods though. There are several things to be taken into consideration. The different feature descriptors need to show some form of diversity in their representations otherwise it will only create multiple calculations for the same information which would risk overfitting \cite{cunningham2000diversity}\cite{krogh1995neural}. A more applicable approach is to use different methods of classification. One can use the Condorcet jury theorem as an example of this methodology, \textit{``If each voter has a probability p of being correct and the probability of a majority of voters being correct is P, then p>0.5 implies P>p. In the limit, P approaches 1, for all p>0.5, as the number of voters approaches infinity''} \cite{cord2008machine}\cite{grofman1983thirteen}.
There is great potential with the use of several different classifiers. \todo{rephrase}The relevant\todo{the relevant space have? what?} space have potential to be much more based on each and\todo{each and every one single one?} everyone single one of the classifiers notation. This because a single classifier  might get a part of the solution in linear case while the use of several can be dimensioned to be able to solve higher dimensional problems. \todo{This is not in line with anything we do. They train on same data but with different learners. all have same targets.}The SVMs should not be over the same datasets since they in that case will have the same errors and thus the SVMs should focus on different targets and create their respective prediction spaces to reduce the possibility the errors are correlated. There are different ways how to \cite{kim2003ensembleSVM}. 


\subsection{Deep support vector machine}
\label{sec:deepSVM}
A \emph{Deep support vector machine} (Deep SVM) is model aimed to enhance the performance by using multiple SVMs by positioning them in layers, inspired by deep belief networks \cite{hinton2006reducing} and other stacking generalization approaches using SVMs \cite{chen2009using}.
The idea is to build layers of SVMs where the output of the previous layer becomes the input of the next layer. The use of more layers give new possibilities in classifications which cannot be achieved with single, however complex, kernel functions. One example of this is the XOR function which can not be solved with a single SVM, but can be when using layers of SVMs. The structure can be perceived in Figure \ref{fig:mltheory:ensemble:deepsvmclassify}, where the initial box, presenting the different feature vectors to the classifying system. The first layer of SVMs receives the feature vectors as input and the output of the first layer becomes the input of the following layer. The number of classifiers in the first layer is only restricted by the number of feature vectors presented to the system, which is an arbitrary number. The final layer, in the figure called ``Meta SVM'', presents the final result of the system. This result is know as the meta-feature vector and has potential of solving higher dimensional problems.

\singlefigure
{figure/classify.png}
{A simplified sketch of how a Deep SVM classifies data. The test data is passed through the first order classifiers and the output of those is the input for the Meta SVM. The output of the Meta SVM is the distance from the decision boundary that the entire classification system has created.}
{fig:mltheory:ensemble:deepsvmclassify}
{1}

The training of a Deep SVM is performed in several steps, as presented in Figure \ref{fig:mltheory:ensemble:deepsvmtrain}. First step is to perform a K-fold split on the training data, $T=\{T_1\cup...\cup T_K\}$, that is applied to the classification system. All the first layer classifiers (first order classifiers) are then trained with the training subset $T_1^c=T\setminus T_1$, in order to use the remaining subset of the training set $T_1$ as a test set. The output of the first order classifiers then becomes the training subset for the second order classifier $T_{1_{meta}}$. This process is repeated K times to produce the full training set for the second order classifier $T_{meta}$. When the K-fold process has been completed the first and second order classifiers can be trained with the full training set $T$ and $T_{meta}$ respectively. The process of training a single SVM is described in Section \ref{sec:mltheory:svm}.  

This setup takes much more time to train than when just using a single SVM. The reward is an estimator that is capable of a more abstract level of classification.

Which kernel functions that the different classifiers have in the Deep SVM does not matter since each unit is independent of the other ones. The selection of kernels in the first order classifiers depend on which feature vectors that they have as input. The classifier at the second layer, however, separates an $n$ number of dimensions if there are $n$ classifiers in the first layer since they have all produced an estimated distance to a decision boundary. Due to the low level of dimensions and the values of the input vector should be positive if a point is predicted correctly, a linear kernel is often possible to apply.

\singlefigure
{figure/train_meta_data.png}
{A simplified sketch of how a Deep SVM is trained. The Meta SVM needs approximations of how the first order classifiers treats the training data in order to fit its own decision boundary. Besides from that the layers can be trained in parallell.}
{fig:mltheory:ensemble:deepsvmtrain}
{1}








