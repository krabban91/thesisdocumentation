% !TEX root = ..\..\main.tex
\section{Classification}

Classification is a general problem in pattern recognition where some form of input value should result in an output value which is representing the label of the element. The research behind classification is extensive and many different fields are working with classification. The choice of which classification method that should be used varies depending on the data. It is of note that no one classifier suits all cases and no one classifier outperforms every other in every other case as per the ``no free lunch'' theorem from Wolpert and Macready \cite{nflTheorem}. The most basic form of classification is binary classification where the data is separated into two categories, for instance as relevant and non-relevant. The input to these algorithms are more commonly known as feature vectors, $\vec{x_i}$. Since the number of dimension of these vectors might vary in-between different feature descriptors, the choice of classifier type is important. Since Support vector machines (SVMs) are good at handling feature vectors of both small and large numbers of dimensions, are fast at classifying and are relatively tolerant to noise, this type of classifier is used in this thesis \cite{kotsiantis2007supervised}\cite{kadam2016study}. 

