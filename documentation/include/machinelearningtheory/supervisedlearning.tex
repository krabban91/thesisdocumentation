\section{Supervised learning}
Supervised learning is a training method for a computer program, where labeled data is explicitly used. Labeled data is a group of samples $\vec{x} \in \vec{X} $ composed of some form of information, or data, distinguishing the samples $\vec{x}_i $ where $i = \{1,...,t\} $ and labels $y_i \in \vec{Y}$, or targets, corresponding each to one sample. The labeled data is usually split into a training set, a validation set and a test set. The training set is the data used to train some form of \todo{Gap in sentence: algorithm, so if presented unknown data, it }algorithm, so if presented unknown data, it should be able to determine which samples belong to which label\todo{check sentance flow}. So presented with a set of samples $\vec{x_j}$ where $j = \{1,...,s\}$ it should predict the correct labels $y_j$. The validation set is then applied to get an idea of how its\todo{it is} performing and to see if further changes is needed to get an acceptable result. If one tries multiple approaches, this should determine which to use if the learning algorithm performs as anticipated. The last part of the labeled data, the test set, is then used to check what a possible expectation of the algorithm could be when presented unlabeled data. When these three\todo{what three parts?} parts are completed the supervised learning is done. 
